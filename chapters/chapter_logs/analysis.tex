\section{Analizowanie logów}
\label{chapter:logs:analysis}

Analizowania informacji zawartych w logach jest operacją skomplikowaną z uwagi na wysokie zróżnicowanie danych, jakie
potencjalnie może być reprezentowane. Opisane w rozdziałach problemy, jakie spotyka się
przy zbieraniu logów z pojedynczego lub rozproszonego systemu informatycznego oraz ich 
późniejszej normalizacji do postaci wspólnej (\ref{chapter:logs:normalize}, \ref{chapter:logs:collecting}), 
przekładają się na jeden wniosek. Ciężko jest przyjąć uniwersalny algorytm, będący w stanie sprostać temu wymaganiu.
Z tego też powodu, to właśnie normalizacja, innymi słowy rozumieniu więcej niż jednego możliwego formatu, jest
szczególnie istotne, jeśli dalsze zrozumieniu, i to w rozumienie zautomatyzowane, jest celem, który chce się 
osiągnąć.

    \subsection{Cele analizowanie logów}
    
    To dlaczego logi są analizowane, zależy silnie od kontekstu w jakim, ta operacja będzie się odbywać. 
    Inne dane będą uzyskiwane z logów systemowych, inne z logów urządzeń sieciowych, a w zupełnie innej
    kategorii znajdą się te informacje, które pochodzić będą od aplikacji działających na systemie. Warto
    nadmienić, że podział, podobny do przedstawionego powyżej, obecny będzie również na poziomie
    kolejnych komponentów systemu i programów. Pomijając jednak ten fakt, 
    można wyróżnić dwa główne cele, z których wszystkie pozostałe będą wynikać,
    a dla których, analiza logów jest istotna.
    
        \subsubsection{Zrozumienie znanych błędów}
        W przypadku dowolnego programu, można wyróżnić zbiór problemów,
        z którymi aplikacja styka się już na poziomie implementacji,
        testów integracyjnych oraz wczesnego etapu działania. Pomimo, że większość tych błędów jest najczęściej
        naprawiana w trybie natychmiastowym. Niestety nie wszystkie, mają szanse być zaadresowane w ten sposób.
        Wyróżnić można taki ich podzbiór, gdzie błędy znajdują się niejako poza aplikacją i wynikają z użytych
        mediów komunikacyjnych (kolejki, protokół) i ujawniają się jedynie w pewnych okolicznościach.
        Przez taką, a nie inną, charakterystykę są ona ciężkie do wykrycia, tym samym zreprodukowania i naprawy.
        Niemniej, z uwagi na wspomniane specjalne uwarunkowania, które mogą zostać wykryte, właśnie poprzez analizę logów,
        możliwa jest wczesna reakcja. Tym też manifestuje się zrozumienie błędów, które zapisane zostały
        w postaci kolejnych rekordów w pliku z logami. 
        Z drugiej strony, należy wspomnieć, że zrozumienie znanych błędów, może też odnosić się to incydentów
        bezpieczeństwa . Fakt niepoprawnego logowania, który powtarza się wielokrotnie, może wskazywać na próby
        uzyskania nieautoryzowanego dostępu do jakiegoś serwisu. Jeśli są to ciągłe próby, gdzie dwoma, ciągle
        zmieniającymi się zmiennymi, są użytkownik oraz hasła, będzie to najprawdopodobniej próba złamania hasła
        metodą ataku siłowego-słownikowego \cite{logging_log_management}. 
        
        \subsubsection{Rozpoznanie nowych zagrożeń}
        Umiejętność zrozumienia istniejących problemów, które mogą zostać wykryte poprzez manualną lub
        automatyczną analizę logów, jest jedną stroną medalu. Po drugiej strony, znajduje się reagowania
        na wydarzenia, które są nowe i nieznane. Są one najniebezpieczniejsze z uwagi na swoją 
        , jeszcze niezrozumianą, naturę oraz implikacje. Cel, jakim jest analiza dążąca ku wykryciu tego
        typu anomalii, jest jednak trudniejszy do osiągnięcia. Wymaga on dużo bardziej wyrafinowanych narzędzi
        operujących na dużych zbiorach danych. Warto dodać, że analizowane rzędy wielkości mogą oscylować, i 
        nierzadko przekraczać, terabajty \cite{logging_log_management}. 
        
    \subsection{Metody analizy}
    
        \subsubsection{Korelacja danych}
        \label{chapter:logs:analysis:methods:correlation}
        Jest to technika analizowania informacji, która może zostać zastosowana nie tylko do logów. Niemniej
        i tutaj okazuje się ona użyteczna. W ogólnym rozumieniu można ją sformułować w postaci następujących punktów:
        \begin{itemize}
            \item w danym momencie wystąpiło wydarzanie \textbf{A},
            \item po wydarzeniu A, można było zaobserwować wydarzenie \textbf{B}, kilka razy w krótkim odstępie czasu.
        \end{itemize}
        Bazując na tym, możliwe jest, że potencjalnie kolejnym zdarzenie będzie wydarzenie \textbf{C}, o krytycznych
        implikacjach dla działającego systemu. Takimi konsekwencjami mogłoby być na przykład wyłączanie pewnej 
        aplikacji, a tym samym przerwa w działaniu usługi. Niemniej, z tak przedstawionego schematu, wynika kilka cech
        tego podejścia. 
        
        Korelacja danych, jak sama nazwa wskazuje, umożliwia podjęcia odpowiednich kroków, jeśli dwa lub więcej wydarzeń
        nastąpi po sobie. Co ważne, wydarzenie te mogą być potencjalnie wcale ze sobą niezwiązane i pochodzić z różnych
        elementów systemu. Inną cechą, i jednocześnie wadą, jest to, że należy wcześniej wiedzieć, że pewna
        sekwencja operacji, wydarzeń następujących po sobie (od razu lub w pewnym odstępie) czasu, doprowadzić może
        do sytuacji krytycznej. Przykładowo, podmiotem korelacji danych mogłaby być kolejka buforująca 
        komunikację między kilkoma komponentami. Kilka z nich zaczyna zgłaszać, poprzez kolejne rekordy w swoich logach,
        że nie udało się wpisać nowej wiadomości do kolejki. W tym samym momencie, logi wspólnego medium, pokazują, że
        kolejka zaczyna się przepełniać. Dla potencjalnego operatora jest to sygnał, że w najbliższym czasie, może dojść
        do faktycznego przekroczenia limitu maksymalnego liczby nieodczytanych wpisów w buforze. Dodatkowo, wie on, że
        kolejka spróbuje skorygować ten błąd poprzez usunięcie najstarszych elementów w swoim buforze oraz zresetowaniu
        wskaźników dla wszystkich producentów oraz konsumentów. Podczas gdy, dla nadawcy wiadomości nie jest to problem,
        wszyscy lub niektórzy odbiorcy stracili by możliwość odczytania pewnych informacji. Jeśli miałyby one
        krytyczne znaczenie, zaraportowany problem, powstały poprzez skorelowania opisanych wyżej wydarzeń, miałby jeszcze
        dalej idące w skutkach konsekwencje.
        
        Analiza tego typu, czyli dochodzenie, które sekwencje i powiązane ze sobą wydarzenie, niosą
        potencjalnie niebezpieczne skutki, może odnosić się do:
        \begin{itemize}
            \item[skali mikro] - oparte jest na wiązaniu ze sobą wartości pól, które opisują dane wydarzenia lub
            ich zbiory. Całe zdarzenie pozostaje tak samo istotne, ale to konkretne informacje w nich zawarte,
            stają się składnikami równania opisującego korelację między poszczególnymi faktami. 
            \item[skali makro] - jest rozszerzenie \textbf{skali mikro} o pobierania dodatkowych informacji z zewnętrznych 
            źródeł dla wspomożenia procesu wnioskowania \cite{logging_log_management}. 
        \end{itemize}
        
        \subsubsection{Analiza statystyczna}
        Metoda korelacji, jak zostało wspomniane w \ref{chapter:logs:analysis}, jest odpowiednim wyborem, jeśli
        znane są warunki, potrzebne do wystąpienia zdarzenia utożsamianego jako niebezpieczne. Z tego też powodu,
        jeśli niemożliwe jest sprecyzowanie takich wymagań, konieczne jest wykorzystanie narzędzi statystycznych.
        
        Na potrzeby wykrywania zdarzeń niepożądanych, można wyróżnić:
        \begin{itemize}
            \item[częstotliwość] - jest to najprostsza, możliwa do zastosowania technika, której opiera się,
            zgodnie z nazwą, na częstotliwość. Istotne jet więc to, jak często wykrywana jest pewna wartość.
            Zlokalizowanie jej może odnosić się zarówno do powtarzającej się identycznej wartości lub ciągłym
            przekraczaniu zadanego progu. Przykładowo, można to odnieść, w pierwszym przypadku,
            do analizy logów serwera WWW i ciągłych wystąpień tego samego adresu IP jako źródła żądania.
            Dla administratora może być to sygnał właśnie podejmowanej próby ataku DDOS, jeśli dodatkowo
            podobne raporty napływają dla innych adresów IP. Z drugiej strony, ciągłe przekraczania wartości
            maksymalnej lub minimalnej, oznaczać może wysokie i stałe obciążanie procesora lub bardzo
            niski czas odpowiedzi serwera. 
            \item[linia bazowa] - idea tej techniki opiera się na ustaleniu progu lub bazy. Pozwala to
            na określenie poziomu poniżej którego napływające informacje, wskazują na normalne działania
            monitorowanego systemu. Z drugie strony, przekroczenie progu, będzie wskazywać na jego
            anormalnego zachowanie. Niestety, podanie poprawnego modelu progu, oddającego
            specyfikę systemu, jest zadaniem skomplikowanym. Aby ustalić bazę, konieczne jest przede wszystkim
            zebranie dużych ilości danych oraz wsparcie osób, posiadających wiedzę ekspercką z danej dziedziny.
            Ich doświadczenie jest konieczne, aby określić jak, w danym przypadku, objawiają się zachowania
            niepożądane, a co uchodzi za zachowanie normalne. Ważne jest, że po pewnym czasie, to jest
            po pojawieniu się nowych danych, ponownie wyznaczyć linię bazową. Bez tej operacji mogłoby się okazać,
            że nie oddaje ona ciągłej dynamiki systemu i analiza jest niemiarodajna \cite{logging_log_management}.
            \todo[inline]{ehhh....to much bullshit}
        \end{itemize}