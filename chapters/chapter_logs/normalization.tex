\section{Problem różnorodności logów}
\label{chapter:logs:normalize}

    \subsection{Problem różnorodności w logach}
    \label{chapter:logs:normalize:differences}
    Implikacją różnorodności omówionej w \ref{chapter:logs:collecting:source} jest różnorodność, jaką przejawiają logi.
    Różnice między logami mogą być subtelne, bezpośrednio dotyczyć pojedynczego rekordu jak również i całości.
    Subtelną różnicą będzie chociażby inny format zapisu daty kiedy utworzono log. Zapis z jednego źródła można
    reprezentować w postaci \textbf{YYYY-MM-DD}, z kolejnego \textbf{YYYYMMDD}, a jeszcze następny może
    do tego celu wykorzystać znacznik czasowy (timestamp). Innym problematycznym przykładem jest to
    jak reprezentowane są poziomy ważności. \textbf{Syslog} wykorzystuje następujące:
    \begin{itemize}
        \item INFO,
        \item DEBUG,
        \item NOTICE,
        \item WARN,
        \item ERR,
        \item CRIT,
        \item ALERT,
        \item EMERG. \cite{logging_log_management}
    \end{itemize} 
    Inaczej sytuacja wygląda jeśli aplikacja korzysta ze specyficznego, dla języka w którym została napisana, formatu.
    Przykładowo Java nie definiuje poziomów takich jak: \textbf{NOTICE}, \textbf{ALERT}, \textbf{EMERG}, \textbf{CRIT}.
    Idąc dalej, pojawią się różnice w zapisie niektórych z nich. \textbf{ERR}, w Java, zapisywany jest jako \textbf{ERROR}.
    Nie istnieje tu także \textbf{NOTICE}, ale Java definiuje \textbf{TRACE}.
    
    Ostatecznie różnice mogą pojawić się jeśli chodzi o strukturę całego rekordu.
    Podobnie jak w CSV (\ref{chapter:logs:structure:record_structure}),
    kolejność ma znaczenie. Jeśli w języku Python znacznik czasu umieszczony jest na 3 pozycji 
    (licząc od lewej strony), dla języka GoLang może być to
    1 pozycja, a dla języka Ruby - 2. Ponadto warto wspomnieć tutaj o NodeJS oraz jednej z
    popularniejszych bibliotek przeznaczonych do logowania. \textbf{Bunyan} zapisuje do pliku kolejne 
    rekordy w postaci \textbf{JSON}'a. Inne biblioteki mogą do tego samego celu używać również XML'a \cite{log_management_explained}.
    
    \subsection{Różnorodność w logach, a przetwarzanie ich}
    Omówione w \ref{chapter:logs:normalize:differences} różnice to jedynie ułamek potencjalnych problemów, 
    które skutecznie utrudniają przetwarzanie logów zgromadzonych z różnych źródeł. Aby temu przeciwdziałać wymagana jest
    aplikacja, zadaniem której byłaby ich normalizacja. Sprowadzanie do postaci spójnej, w przypadku logów, dzieli się
    na operacje, które rozwiązać można w sposób wspólny dla wszystkich formatów, jak i takie, dla których konieczne jest
    oddelegowanie specjalnie zaprojektowanego do tego celu procesora (aplikacji transformującej). 
    Za operacje wspólne uznać można:
    \begin{itemize}
        \item[\textbf{wykrywanie poziomu ważności}] - jest to operacja, którą najłatwiej rozwiązać przy pomocy wyrażeń regularnych.
        Przygotowane wyrażenie musi uwzględnić wszelkie możliwe wariacje pojedynczego poziomu. Niestety jest to
        rozwiązanie bardzo kosztowne pod względem zużycia mocy obliczeniowej procesora. Wynika to ze sposobu w jaki działają 
        wyrażenia regularne. Instynktownie każda możliwa wartość poziomu ważności oraz jej warianty zapisane zostałyby 
        jako alternatywy. RegExp podczas przeszukiwania podanego ciągu znaków musi przejrzeć cały łańcuch znakowy dla każdej z nich.
        Im więcej alternatyw oraz im dłuższe jest zdanie, które musi zostać przeanalizowane, tym więcej czasu jest wymagane.
        W przypadku dużych wiadomości, takich jak te generowane z niektórych komponentów OpenStack, może się okazać, że 
        pewne wiadomości będą tracone. Rozwiązaniem tego problemu jest założenie, że poziom ważności znajduje się w k-początkowych
        znakach łańcucha tekstowego. W większości przypadków będzie to podejście prawidłowe i czas wykrycia może się znacząco zmniejszyć.
        Powołując się jednak na problem różnorodności może dojść do sytuacji, w której omawiana wartość zostanie umieszczona
        poza przeszukiwanym ciągiem lub część będzie się w nim znajdować, a część nie. W tym wypadku jedynym wyjściem jest uznanie, że
        wartość jest nieznana. Aplikacja może także nie zapisywać poziomu ważności.
        
        Dużo bardziej stanowcze rozwiązanie to wymuszenie odpowiedniego formatu logu. Z punktu
        widzenia późniejszej normalizacji, jest to rozwiązanie najbardziej efektywne, ponieważ wiedza
        na temat lokalizacji konkretnych wartości jest znana wcześniej.
        
        \item[\textbf{czas wygenerowania logu}] - jest to wartość mniej istotna, jeśli chodzi o normalizację, szczególną rolę odgrywa natomiast
        w trakcie analizowania logów. Jest jednak z tego właśnie powodu wyjątkowa i proces normalizacji powinien zapewnić, że będzie ona reprezentowana w spójnej postaci.
        Nie zawsze jednak jest to możliwe. Problemem tutaj, oprócz
        różnych formatów, w jakich znacznik czasowy może być zapisany, jest rozbieżność zegarów systemowych na maszynach,
        z których log pochodzi. Potencjalnie, w 9 na 10 komputerów, zegar systemowy może podawać czas ze wskazaniem
        na strefą czasową UTC. Kolejna maszyna nie musi jednak podążać tym torem, a strefa czasowa może być ustawiona
        na k-godzin w przód lub w tył od strefy UTC. Nie jest to problem jeśli biblioteka użyta do logowania aktywności
        danej aplikacji zapisuje kolejne rekordy w uniwersalnej strefie czasowej. 
        Ponadto zegary systemowe, mimo używania tej samej strefy czasowej, mogą wykazać nawet kilkuminutowe różnice. 
        Program odpowiedzialny za normalizację kolejnych rekordów napływających z całego systemu może jedynie
        obejść tę przypadłość, dodając kolejny znacznik czasowy, wskazując na moment odebrania logu. 
        Nie jest to jednak rozwiązanie idealne. Najczęściej dane wysyłane są przez sieć, która w jednym punkcie może wykazać chwilowo
        zwiększony ruch, a w innym zmniejszony. Skutkiem takiego zbiegu wydarzeń może być fakt, że wydarzenie A miało miejsce
        chwilę przed wydarzaniem B, podczas gdy, tak naprawdę, wydarzenia B nastąpiło kilka minut po zdarzeniu A. 
        
        \item[\textbf{rekordy w wielu liniach}] - najczęściej pojedynczy wiersz odpowiada pojedynczemu rekordowi, jako logicznie
        spójnej całości. Często jednak sytuacja wygląda odwrotnie. Pojedyncze wydarzenie zostaje zapisane w pliku w kilku
        lub kilkunastu liniach. Dla administratora przeglądającego taki log manualnie nie jest to problem. Identycznie powinien on
        móc zobaczyć ten sam log w tej samej postaci. Problem wielu linii jest jednak inny dla każdego spotkanego formatu logu.
        W tym wypadku konieczne jest ustalenie co rozpoczyna lub kończy wpis znajdujący się w więcej niż jednej linii. 
        
    \end{itemize}
    
    