\section{Log - historia oraz znaczenia}
\label{chapter:logs:history}

    \subsection{Historia}
    Logi, czyli wydarzenia związane z działaniem aplikacji, pojawiły się prawdopodobnie razem z pierwszymi programami.
    Jednym z najbardziej znanych przykładów oprogramowania, które zostało stworzone specjalnie z myślą o logach, jest \textbf{syslog}
    \footnote{Syslog \url{http://www.syslog.org/}}. Narzędzie powstało w roku 1980, początkowo zaprojektowane specjalnie
    dla aplikacji \textbf{Sendmail}\footnote{Sendmail - \url{https://www.sendmail.com/sm/open_source/}}. Od tamtego momentu,
    doczekało się wielu implementacji, aż w końcu zostało uznane za standard w roku 2009\footnote{RFC-5424 \url{https://tools.ietf.org/html/rfc5424}}.

	\subsection{Czym jest log ?}
    \label{chapter:logs:history:what_is_log}
	W informatyce log jest nieodłącznym elementem każdej, dobrze napisanej, aplikacji. 
    W przypadku takiego programu jest to narzędzie dzięki któremu informuje on
    o kondycji swojej, lub systemu \footnote{Dobrym przykładem jest tutaj \textit{/var/log/messages} w systemach Unix, 
    gdzie zapisywane są wydarzenia pochodzące z całego systemu, a nie tylko z konkretnej aplikacji}, potencjalnego obserwatora.
    Z tego też powodu log nie jest faktycznym wynikiem, wyjściem pewnego algorytmu. Niemniej nie umniejsza to jego znaczenia.
    Logi \textit{access.log} serwera WWW Tomcat stanowią doskonałe poparcie tego stwierdzenia. 
    Wewnątrz może odnaleźć informacje o każdym żądaniu odebranym przez daną instancję serwera \cite{tomcat_logs}.
    
    \subsection{Znaczenie logów w procesach biznesowych oraz administracyjnych}
    Zgodnie z paragrafem \ref{chapter:logs:history:what_is_log}, log często jest elementem systemu informatycznego, 
    który pozwala na nie tyle lepsze jego zrozumienie, ale także reagowanie na pewne wydarzenia wyjątkowego.
    Z punktu widzenia administratorów systemu dane zawarte w logach serwera WWW pozwalają na określenie 
    sekwencji żądań wysłanych do serwera i mogą na przykład przysłużyć się wykryciu ataku sieciowego.
    Atak sieciowy, z definicji, jest atakiem przeprowadzonym z użyciem sieci komputerowej.
    Jednym z jego przykładów jest atak typu \textbf{DDOS} - \textit{Distributed Denial Of Service}. 
    Główną jego cechą jest wysłania tak wielu żądań HTTP do serwera, z możliwe jak największej liczby klientów,
    celem zajęcia wszystkich zasobów. Jedną z metod wykrywania takiego ataku, jest zbierania i analiza logów
    w czasie rzeczywistym. Wiedza na temat zwiększanego ruchu sieciowego oraz dane zebrane z zasobów systemowych (takie jak zużycie pamięci lub czasu procesora)
    dają możliwość wczesnego wykrycia takiego ataku \cite{web_based_attacks}.
    
    Analiza zachowań użytkowników jest kolejnym z elementów do którego zastosować można logi.
    Kontynuując wątek logów \textit{access.log}. Na podstawie analizy żądań HTTP można wywnioskować, które z zasobów 
    serwera są najczęściej pobierane, otwierane przez użytkowników konkretnej aplikacji. Szczególnie
    przydatna taka wiedza może okazać się w przypadku stron internetowych o profilu użytkowym.
    Większość klientów serwisu zainteresowanych jest materiałami odnoszącymi się do wiadomości z ostatnich 24h,
    użytkownicy przede wszystkim wykorzystują ten serwis do czytania materiałowych prasowych lub ściągania plików.
    Wiedza na taki temat pozwala na lepsze dostosowania zawartości strony do potrzeb odbiorcy.
    