\section{Klasyfikacja logów}
\label{chapter:logs:classification}

Logi to więcej niż tylko aktywność aplikacji zapisana do pliku lub wysłana przez sieć.
Wyróżnia się wśród nich logi \textbf{systemowe}, \textbf{sieciowe} oraz generowane przez \textbf{aplikacje}.

    \subsection{Logi systemowe}
    \label{chapter:logs:classification:system}
    Są to wszelkie wydarzenia wygenerowane przez system operacyjny, jego jądro lub 
    wbudowane w niego aplikacje. Z tego powodu pod tą kategorią znajdują się między innymi:
    \begin{itemize}
        \item logowanie oraz wylogowanie użytkownika,
        \item uruchomienie oraz wyłączenie samego systemu,
        \item wydarzenia związane z urządzeniami peryferyjnymi,
        \item problemy z system plików oraz urządzeniami dyskowymi.
    \end{itemize}
    Logi tego rodzaju są momentami najbardziej wartościowe, z punktu widzenia administratora
    danego serwera. Pozwalają one, niejednokrotnie w połączeniu z logami sieciowymi oraz 
    aplikacji, poznać przyczynę błędu. Te informacje są nieodzowne w usuwaniu usterki
    oraz zrozumieniu jej przyczyn, co pozwala na zabezpieczenie systemu przed kolejnymi
    problemami tego samego typu i o podobnym podłożu \cite{logging_log_management}.
    
    \subsection{Logi sieciowe}
    \label{chapter:logs:classification:network}
    Są one generowane przez urządzanie sieciowe, często o krytycznym znaczeniu dla całej infrastruktury. 
    Zaliczają się tu na przykład router'y czy też firewall'e. Ciągła analiza wydarzeń, wygenerowanych przez 
    wspomniane elementy infrastruktury sieciowej, pozwala na wykrywanie ataków sieciowych
    czy też każdej, anormalnej sytuacji. Możliwa jest również aktywna bądź pasywna 
    optymalizacja tej struktury \cite{logging_log_management}. 
    
    \subsection{Logi aplikacji}
    \label{chapter:logs:classification:application}
    Źródłem tej klasy logów są programy działające na maszynach, pod kontrolą konkretnego
    systemu operacyjnego, wykorzystujące sieć jako medium transmisji danych. Są to
    aplikacje dostarczające faktycznej funkcjonalności swoim użytkownikom. Z tego powodu
    logi aplikacji są tak samo ważne jak logi systemowe oraz sieciowe \cite{logging_log_management}. 
    
    \subsection{Logi, a bezpieczeństwo}
    \label{chapter:logs:classification:security}
    Logi związane z bezpieczeństwem wliczają się w 3 kategorie omówione powyżej (\ref{chapter:logs:classification:system},
    \ref{chapter:logs:classification:network}, \ref{chapter:logs:classification:application}). Jednak
    ich znaczenie to osobna kategoria, ponieważ incydenty związane z bezpieczeństwem, mogą wydarzyć się
    na każdym poziomie. Tego typu zdarzenia związane będą z następującymi, potencjalnymi sytuacjami:
    \begin{itemize}
        \item[\textbf{nieautoryzowany dostęp}] - poprzez tworzenie zapisu każdej próby dostępu do serwera, udanej lub nie, maszyny lub aplikacji możliwe jest wykrywanie nieautoryzowanego dostępu.
        \item[\textbf{oprogramowanie antywirusowe}] - może zapisywać, oprócz logów związanych z jego normalną aktywnością
        (aktualizacja bazy danych), także wydarzenia, których źródłem było wykrycie złośliwego oprogramowania.
        \item[\textbf{żądanie serwera}] - odnosi się zarówno do serwerów aplikacji, proxy oraz do serwerów
        obsługujących zewnętrzne logowanie (OpenID, LDAP). 
        
        Oprócz możliwości wykrycia nieautoryzowanego dostępu logi mogą zawierać także listę wszystkich żądań. Jest to szczególnie przydatne dla serwerów
        typu proxy, których jednym z przypadków użycia, jest ograniczanie dostępu do wyszczególnionych zasobów
        sieciowych. Serwer jest w stanie odpowiedzieć na pytanie administratora, w którym pyta on, kto  i w jakich
        godzinach próbował wykonać żądanie do strony WWW \textbf{http://bad.site.com}, która zgodnie z
        polityką przedsiębiorstwa, jest zablokowana.
    \end{itemize}