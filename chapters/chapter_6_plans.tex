\chapter{Plany rozwoje lub obecnie implementowane}
\label{chapter:application_own:plans}

Aplikacja do zarządzania logami jest w fazie aktywnego rozwoju. 
Do tej pory udało się osiągnąć relatywnie prostą architekturę, dzięki której
możliwe jest zbieranie logów ze wskazanych maszyn oraz opisanie ich
dodatkowymi danymi, pozwalającymi łatwo zidentyfikować skąd i kiedy
zostały pobrane oraz wskazać na ich znaczenie w kontekście monitorowanego systemu.
Przedstawione poniżej elementy są na obecną chwilę rozwijane i dotyczą aplikacji, programów
oraz zagadnień, które znajdują w zakresie przetwarzania logów. Niemniej, poza tym, autor
pracy dyplomowy, bierze oraz brał aktywny udział w tworzeniu, projektowaniu i dyskusjach
dotyczących innych części końcowego produktu. Z uwagi na zawodowy charakter opisywanych rozwiązań nie
jest możliwe zaprezentowanie oraz omówienie detali na odpowiednim poziomie szczegółowości. 
Niemniej dotyczą one zagadnień takich jak:
\begin{itemize}
    \item tworzenie klastrów dla poszczególnych aplikacji całego systemu,
    \item automatyzacji powyższych operacji,
    \item \textbf{multi-tenancy} w postaci wtyczki dla aplikacji \textbf{Kibana},
    \item przenoszenia, poza \textbf{monasca-log-api}, kolejnych komponentów do języka Python,
    \item korelowania metryk oraz informacji zebranych z logów.
\end{itemize}

\section{Alarmy a logi}
\label{chapter:application_own:plans:alarm_on_logs}

Jednym z założeń projektu \textbf{monasca} jest powiadamianie użytkownika o 
sytuacjach krytycznych zaistniałych w systemie. Funkcjonalność realizowana jest na podstawie
metryk zbieranych z całego systemu. Przekroczenie założonych przez administratora wartości
krytycznych, skutkuje wygenerowaniem alarmu. Jednak idea ta, nie może zostać
tak łatwo przełożona na logi. Problemem jest przedstawienie logów w postaci metryk - czyli wartości
numerycznych. W przeciwieństwie do danych dotyczących chociażby zużycia przestrzeni dyskowej lub podlegających większej
fluktuacji, obciążeniu procesora, logi stanowią element zbioru metryk nieciągłych \footnote{Sparse metric - metryki okresowe,
zwracające wartości w nieregularnych odstępach czasu}. 

Częścią rozwiązania problemu byłaby modyfikacja aplikacji \textbf{monasca-thresh}, odpowiedzialnej za faktyczne generowanie 
alarmów. Na obecną chwilę, jeśli dla danej metryki, brak jest danych, monitor zdefiniowany dla danego alarmu powoduje jego 
wejście w stan \textbf{UNDETERMINED}. W przypadku logów nie jest to jednak sytuacja oczekiwana. Przykładowo brak wystąpień 
błędów dla danej aplikacji, powinien faktycznie oznaczać, że aplikacja lub system pracuje poprawnie i potencjalny operator nie 
ma powodów do niepokoju. Innymi słowy, odpowiadałoby to stanowi \textbf{OK}. Dopiero co najmniej jedna wartość wykraczająca 
poza ramy danego alarmu, powinna spowodować jego przejście w stan wysoki oraz wygenerowanie powiadomienia lub powiadomień.

Funkcjonalność opisana powyżej jest obecnie w fazie projektowania i podlega dyskusji między członkami całej społeczności.
Została ona omówiona podczas \textbf{Monasca Mid-Cycle}, spotkania wszystkich programistów zrzeszonych wokół projektu
\textbf{monasca}, które odbywało się 3 oraz 4 lutego 2016 roku. Autor poniższej pracy dyplomowej był aktywnie włączony
w dyskusję, gdzie następujące aspekty zostały omówione:
\begin{itemize}
    \item wprowadzenie nowej zmiennej \textbf{period} do metryk opisujących jak długi okres czasu musi upłynąć, aby
    zmienił się stan alarmu powiązanego z metryką. Dla omawianej koncepcji (alarmy a logi) przyjęte zostało użycie wartości
    \textbf{-1} aby wskazać, że dana metryka jest nieokresowa,
    \item sposób dostarczenia nowych metryk do systemu \textbf{monasca}. Miałyby one trafiać, poprzez kolejkę Kafka,
    do aplikacji \textbf{monasca-thresh},
    \item modyfikacji \textbf{monasca-thresh}, aby rozumiał metryki okresowe oraz nieokresowe.
\end{itemize}
\newglossaryentry{healthcheck}{
    name={Healthcheck},
    description={
        Specjalny adres, dostępny w \textbf{REST}'owym API, który informuje odpytującego o stanie aplikacji. Zależnie
        od implementacji, można wyróżnić co najmniej dwa typy tego rodzaju komponentów:
        \begin{itemize}
            \item[proste] - informujące jedynie o tym, czy API jest dostępne i można wykonywać żądania,
            \item[złożone] - oferujące tą samą funkcjonalność co \textit{proste} oraz, dodatkowo,
            weryfikujące stan zależnych serwisów, jak chociażby baza danych.
        \end{itemize}
    }
}
\newglossaryentry{loadbalancer}{
    name={Load Balancer},
    description={
        Rodzaj aplikacji, najczęściej znajdujący się przed klastrem serwerów. Jego nadrzędnym zadaniem jest
        dystrybuowanie ruchu do poszczególnych węzłów według zadanego algorytmu. Operacja ta ma na celu,
        równomiernie rozłożenie obciążenia, które generowane jest przez klientów, łączących się do klastra
    }
}

\subsection{Healthcheck}
\label{chapter:application:plans:healthcheck}

Mechanizmy oferowane przez \glslink{healthcheck}{\textbf{healtcheck}'i} dostępne są, na obecną chwilę, jedynie
w następujących aplikacjach: \textbf{monasca-api} oraz \textbf{monasca-log-api}. Dodatkowo, są to wersje napisane w języku
Java. Biorąc pod uwagę trend w społeczności \textbf{monasca} dążący do przejścia na wersje napisane w języku Python,
autor, poniższej pracy dyplomowej, stworzył propozycję implementacji na potrzeby \textbf{monasca-log-api}. Dzięki
temu, możliwe jest wysyłanie żądań do serwera, na które odpowiedzieć on może w następujący sposób:
\begin{itemize}
    \item[HTTP 204] - oznacza, że \textbf{monasca-log-api} działa oraz \textbf{Kafka}, które wykorzystywane jest
    do przysyłania danych do \textbf{monasca-log-transformer} jest również obecne w sieci, a wymagane \glslink{kafka_topic}{topic'i} są
    utworzone
    \item[HTTP 503] - komponent zależny - kafka - jest niedostępna lub niepoprawnie skonfigurowana,
    \item[HTTP 404] - \textbf{monasca-log-api} jest niedostępny
\end{itemize}

Całość została oparta o bibliotekę \textbf{oslo.middleware}. Warto dodać, że ta oraz inne biblioteki, których nazwa
rozpoczyna się od \textbf{oslo.} są, praktycznie, nieodłączną częścią chmury obliczeniowej \textbf{OpenStack}. Z tego też powodu,
są wykorzystywane w bardzo dużej liczbie projektów i to nie tylko tych, które stanowią główną część \textbf{OpenStack} (jak chociażby
\textbf{Horizon czy \textbf{Nova}}).

Dodatkowo autor pracy dyplomowej, zajął się zweryfikowaniem możliwych alternatyw dla omawianego
rozwiązania. Problematyczna okazuje się, bowiem, jedna różnica między wersjami napisanymi w językach \textbf{Python} oraz \textbf{Java}.
Implementacja oparta o język \textbf{Java} wykorzystuję bibliotekę \textbf{DropWizard}, która domyślnie wspiera funkcjonalność określaną
nazwą \textit{healthcheck}. Dodatkowo, serwer uruchamia kolejną (wbudowaną) usługę dostępną pod innym portem.
Jest to duża zaleta w momencie kiedy \textbf{API} umieszcza się wraz z innymi jego instancjami, a gdzie wszystkie umieszczone są
za specjalnym koordynatorem ruchu - \glslink{loadbalancer}{\textbf{loadbalencer}'em}. Posiadanie \textbf{healtcheck} poza głównym API, pozwala
na wydzielenie tych żądań spoza zakresy pracy \textbf{loadbalencer}. Niestety, biblioteki dostępne w \textbf{Python} nie dają takich możliwość.
Również próby uruchomienia wbudowanego serwera WSGI, działającego na innym porcie, okazały się nieskuteczne. Dużym problem okazał się
sposób pracy serwera \glslink{wsgi}{WSGI} wybranego przez społeczność - Gunicorn. W momencie załadowania kompletnej aplikacji, Gunicorn
wchodzi w pętlę, której nadrzędny wątek jest blokujący i wznawiany w momencie otrzymania żądania. Przez to uruchomiony wewnętrznie serwer, nie
pozwala na uruchomienie właściwej aplikacji. Rozwiązaniem wydawałoby się więc wystartowanie go wywnętrz oddzielnego procesu, działającego równoległe
do głównego wątku API. Niestety, przy tym podejściu, niemożliwe jest poprawne wyłączenie zarówno wątku \textbf{healthcheck} oraz \textbf{API}.
Biorąc pod uwagę te problemy oraz możliwości oferowane przez \textbf{oslo.middleware}, omawiana funkcjonalność została zrealizowana jako filtr,
uruchamiany przed faktyczną logiką biznesową. 

\todo[inline]{po zaakceptowanie zmiany muszę do przenieść do kontrybucji}

Propozycja implementacji dostępna jest pod adresem \url{https://review.openstack.org/#/c/249685/} i oczekuje na akceptację 
społeczności lub dalsze usprawnienia.
\section{Wiele logów dla monasca-log-api (bulk-mode)}
\label{chapter:application_own:plans:bulk_monasca_log_api}

W pierwszych słowach warto nadmienić, że opisana w poniższym rozdziale funkcjonalność oryginalnie została 
zaproponowana zanim zespół, którego autor pracy dyplomowej jest członkiem, rozpoczął właściwie prace
nad implementacją kolejnych elementów, wchodzących w skład całego systemu. Na obecną chwilę, z ideą 
dodania trybu \textbf{bulk} do \textbf{monasca-log-api}, wyszła firma \textbf{TSV}, która zauważyła brak tej
ważnej funkcjonalności. Interesującym jest, że ten sam fakt, autor oraz inni członkowie zespołu, zauważyli
w tym samym momencie. 

Ideą całego rozwiązania będzie możliwość przesyłania z \textbf{monasca-log-agent} wielu zebranych logów
w jednym żądaniu. Poprzez to cały proces przetwarzania zostanie przyspieszony i ostatecznie logi
będzie można analizować szybciej. Szybsza analiza wiąże się również z szybszym zapisaniem logów w centralnej lokalizacji,
a tym samym zabezpieczeniem całego systemu przed ich utratą. Autor pracy dyplomowej jest jedną z dwóch osób, które będą
współpracowały z firmą TSV na wymienionych poniżej płaszczyznach:
\begin{itemize}
    \item akceptacja zaproponowanego przez TSV planu implementacji,
    \item zaproponowanie koniecznych zmian, które należy wprowadzić do istniejącej implementacji \textbf{monasca-log-api}
    w kontekście adresów pod którymi dostępna będzie logika, a na które będzie można wysyłać logi,
    \item ocenie i testowaniu zmian w kodzie, dążących do rozszerzenia \textbf{monasca-log-api},
    \item finalnej akceptacji zmian.
\end{itemize}
\section{Monitorowanie Kibana}
\label{chapter:application_own:plans:monitoring_kibana}

Od wersji 4.2 program Kibana dostarcza możliwości weryfikacji stanu serwera. Jest to funkcjonalność znana pod ogólną nazwą
\textbf{healthcheck}. Najbardziej istotną korzyścią wynikającej z wprowadzenia tego typu funkcji jest 
udostępnienie informacji, przedstawionych na rysunku \ref{chapter:application_own:plans:monitoring_kibana:picture},
poprzez tradycyjne wywołania HTTP. 

\begin{figure}[H]
    \centering
    \includegraphics[width=1.0\textwidth]{images/kibana_status}
    \caption[Status serwera Kibana]{
        Status serwera Kibana, źródło: opracowanie własne
    }
    \label{chapter:application_own:plans:monitoring_kibana:picture}
\end{figure}

Autor, poniższej pracy dyplomowej, zaproponował wprowadzenie do projektu \textbf{monasca-agent} nowych komponentów, które
wspierałyby okresowe zbierania metryk z programu Kibana. Pobrane informacje mogłyby służyć do tworzenia
nowych alarmów lub historycznej analizy stanu aplikacji z wykorzystaniem diagramów dostępnych z poziomu komponentu 
\textbf{monasca-ui}. Całość implementacji zawarta byłaby w dwóch, nowych, elementach \textbf{monasca-agent}, które
pozwalałyby na:
\begin{itemize}
    \item[automatyczną detekcję Kibana] - podczas startu \textbf{monasca-agent} uruchamiany jest specjalny algorytm, którego
    celem jest automatyczne wykrycie działających aplikacji. Detekcja odbywa się na tej samej maszynie na której działa 
    \textbf{monasca-agent}. W przypadku stwierdzenie, że wspierane aplikacje są uruchomione, \textbf{monasca-agent} tworzy 
    pliki konfiguracyjne informujące proces kolektora, że powinien on uruchomić dedykowany mini program, który w wyniku
    swojego działania, utworzył by nowe metryki.
    \item[pobieranie metryk] - okresowo Kibana byłaby odpytywana o aktualny stan. Uzyskany dane byłyby transformowane
    na metryki.
\end{itemize}