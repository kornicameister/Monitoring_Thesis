\chapter[Wstęp]{Wstęp}
\label{chapter:introduction}

\section{Uzasadnienie i wybór tematu}

\textbf{Zbieranie, analiza oraz wnioskowanie - znaczenie logów w chmurach obliczeniowych}. 

Chmury obliczeniowe są obecnie w fazie, która w cyklu życia żywych organizmów, określana jest mianem gwałtownej ekspansji. Wiele firm
informatycznych takich jak HP, organizacja stojąca za systemem operacyjnym RedHat inwestują swój czas w tematykę związaną z chmurami.
Najczęściej wynikiem takich działań jest produkt końcowy - dystrybucja OpenStack według przepisu danej firmy. Ciekawostką jest tutaj,
że, najbardziej obecnie popularne, rozwiązanie, jakim jest OpenStack, to produkt o otwartym kodzie źródłowym. Podobnie zresztą, jak
aplikacje i biblioteki programistyczne, które powstały, i powstają dalej, jako produkt uboczny prac nad poszczególnymi komponentami
wspominanej implementacji chmury obliczeniowej. Ten fakt, jest nade wszystko źródłem popularności i powodem, dla którego duże spółki, 
korporacje informatyczne delegują zespoły programistów aby pracowały nad rozszerzaniem kodu źródłowego, czy to w kontekście
dodawania nowych funkcji, czy też naprawy zauważonych błędów. Na obecną chwilę (styczeń 2016), na stronie \href{https://wiki.openstack.org/wiki/Contributors/Corporate}{\textbf{OpenStack Contributors}}\footnote{https://wiki.openstack.org/wiki/Contributors/Corporate}
znaleźć można informację o ponad 310 firmach, które oficjalnie podpisały umowę o współpracy z fundacją OpenStack. Liczba, faktycznych kontrybutorów,
jest jednak jeszcze większa, ponieważ wspomniana strona nie specyfikuje osób indywidualnie zaangażowanych w rozwój \textbf{OpenStack}.
Przedstawiony powyżej rozmiar całej społeczności, przekłada się w sposób bezpośredni na szybki rozwój i szybkie adoptowania chmury przez firmy, chcące wykorzystywać ją do realizowania własnych celów biznesowych. Szybko rozwój, stanowił także bodziec lub spopularyzował usługi takie jak
\textbf{PaaS}\footnote{Platform As a Service},
\textbf{IaaS}\footnote{Infrastructure As a Service},
\textbf{SaaS}\footnote{Software As a Service}.
Ich wspólnym mianownikiem jest natychmiastowa dostępność, odpowiednio, platformy, infrastruktury lub oprogramowania, zaraz po zakupie. Ponadto wszystko
odbywa się na wirtualnej płaszczyźnie z minimalnym wkładem własnym, ze strony kupującego. Zakres usług oferowanych w modelu \textit{as a service}, ciągle
się rozszerza. Rozwój odbywa się, zarówno, w płaszczyźnie poziomej, gdzie kolejny aspekty związane z oprogramowaniem, można wykupić, podobnie do karnetu 
na siłownię, jak i płaszczyźnie pionowej, gdzie coraz to lepsze aplikacje tworzone są, aby zastąpić stare. 
Temat pracy dyplomowej, który można by przedstawić, po angielsku, jako \textbf{Logging As A Service}, sam w sobie nie jest zagadnieniem
nowym. Niemniej, został on zaproponowany jako uzupełnienie dla projektu \textbf{monasca}\footnote{Nadrzędnym i pierwszym kontrybutorem była firma HP} przez
firmę Fujitsu, której autor pracy dyplomowej jest pracownikiem. 
Podczas, gdy \textbf{monasca} skupia się na zagadnieniu monitorowania aplikacji oraz maszyn, na których te aplikacje działają. Niemniej, informacje
zebrane przez \textbf{monasca} są zobrazowaniem stanu w jakim znalazł się pewien komponent w danej chwili. Oprócz tych informacji, aplikacje produkują
również logi. Nawiązując do tematu, okazuje się, że operacje takie jak zbieranie logów z dużego systemu informatycznego, ich przechowywanie
czy tez w dalszej kolejności przetwarzanie ich na potrzeby analizy, jest dużo bardziej skomplikowane, niż mogłoby się wydawać. 
Ponadto, źródło informacji, jakimi są kolejne rekordy zapisywane w plikach logów, często okazuje się nieocenione w odnajdywaniu
pierwszych symptomów, które mogą oznaczać poważne problemy w poprawnym działaniu monitorowanych aplikacji.

\todo[inline]{Czy odniosłem się do tematu w tym wstępie?}

\section{Cel i zakres pracy}

Głównym celem pracy magisterskiej jest zaprezentowanie rozwijającego się trendu \textbf{monitorowania jako serwisu} dla usług działających w chmurach,
ze wskazaniem na \textbf{logowanie jako serwis}. Pomimo, że \textbf{LaaS}\footnote{Logging as a Service - logowanie jako serwis} 
może być traktowane jako całkowicie odrębny segment, logicznie jest on częścią monitorowania i możliwości przezeń oferowane wpisują się w ten sam
zbiór funkcjonalności. 
Ponadto, uwago zostanie zwrócona również na aspekt pracy dyplomowej związany z otwarto źródłowym charakterem aplikacji oraz
współpracy autora z zespołem programistów, zatrudnionych w firmie Fujitsu oraz tymi, którzy należą do, ciągle rosnącej, społeczności zrzeszonej wokół projektu \textbf{monasca}. 

Ponadto, poprzez przedstawienie obecnej architektury oraz serwisów, wchodzących w skład całego rozwiązania, autor przedstawi złożoność projektu
oraz swój osobisty wkład w jego dalszy rozwój, jak i w to, co do tej pory zostało już zaimplementowane. Warto dodać, że całość
aplikacji tworzona jest na wielu płaszczyznach. Nadrzędnym aspektem jest praca zawodowa, którą autor pracy dyplomowej, wykonuje.
Ważnym elementem jest także współpraca ze społecznością całego projektu, celem akceptacji proponowanych zmian
oraz dyskusji, jak powinny wyglądać elementy, które w całym projekcie nie były jeszcze obecnie. Ostatecznie, celem autora 
jest zaproponowanie nowej funkcjonalności w ramach pracy dyplomowej.

Ostatecznie, celem autora pracy, poprzez omówienia monitorowania oraz logowania w modelu chmury obliczeniowej, jest zdobycie wiedzy
teoretycznej, pozwalającej na lepsze zrozumienia tworzonego rozwiązania. Tym samym, autor zdobędzie potrzebne umiejętności i informacje, 
które pozwolą na lepsze i rzeczowe komentowanie, omawianie oraz, ostatecznie także i, proponowanie nowych funkcji oraz rozwijanie
już istniejących.  