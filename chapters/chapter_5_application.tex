\chapter{Część praktyczna - wkład własny}
\label{chapter:application_own}

\newglossaryentry{keystone}{
    name={Keystone},
    description={Aplikacja o kluczowym znaczeniu dla chmury obliczeniowej \textbf{OpenStack}. Dzięki niej możliwe jest tworzenia nowych użytkowników, zarządzania ich uprawnieniami oraz ostatecznie autentykacja i autoryzacja}
}
\newglossaryentry{frontend}{
    name={Frontend},
    description={odnosi się, w przypadku aplikacji komputerowych, do tych ich części, która jest bezpośrednio widoczna dla użytkownika końcowego. Z nią też, użytkownik wchodzi
        w interakcje rozumiane jako akcje. Bezpośrednim ich skutkiem jest komunikacja z warstwą \textbf{backend'u}, przetwarzającą żądania i zwracającą wynik. Rezultaty są później, prezentowane
        użytkownikowi, przykładowo, w postaci tabeli lub okna dialogowego}
}
\newglossaryentry{backend}{
    name={Backend},
    description={w najprostszym tego słowa rozumieniu, znajduje się za częścią
        wizualną aplikacji (\textbf{frontend}) i w oczywisty sposób nie jest bezpośrednio widoczny dla
        użytkownika. Innymi słowy, nie musi wcale wiedzieć o jego istnieniu lub znać jego poziomu złożoności.
        Zadaniem, tej warstwy, jest odebrania żądania, przetworzenie go i zwrócenie wyniku, nie zajmując
        się szczegółami związanymi z graficzną wizualizacją rezultatów}
}
\newglossaryentry{node_js}{
    name={NodeJS},
    description={oparty o silnik \textbf{Chrome V8}, oferuje rozwiązanie pozwalająca na uruchamiania kodu \textbf{JavaScript} poza jej tradycyjnym
        środowiskiem, a mianowicie, jako pełnoprawny program, podobny między innymi do \textbf{Google Chrome} czy też \textbf{Notatnika}. Jego
        dewizą jest \textbf{JavaScript on Server}. Dzięki niemu możliwe jest tworzenia kodu serwera w języku originalnie zaprojektowanym
        do wzbogania interfejsu użytkownika na stronach internetowych, podobnie jak to jest czynione w językach takich jak \textbf{Java} lub \textbf{Python}}
}
\newglossaryentry{wsgi}{
    name={WSGI},
    description={Specyfikacja dla języka Python opisująca komunikacją między serwerem a aplikacją. 
        Stanowi zbiór wytycznych oraz reguł realizujące podwaliny spójnej bazy dla tworzonych programów.
        Dzięki temu większość aplikacji, napisanych dla WSGI, jest w stanie współpracować, bez żadnych lub
        z małymi modyfikacjami, z większością implementacji WSGI
    }
}
\newglossaryentry{orm}{
    name={ORM},
    description={
        Mapowanie obiektowo-relacyjne jest techniką według, której możliwe jest reprezentowanie bazy danych pod postacią obiektów.
        Inną nadrzędną cechą jest możliwość współpracy z więcej niż jednym silnikiem bazodanowym bez konieczności,
        lub z małymi zmianami, bazy kodu.
    }
}

Autor, poniższej pracy dyplomowej, zaangażowany był w tworzenia modyfikacji dla praktycznie wszystkich z omawianych powyżej komponentów i w przyszłości będzie kontynuował pracę związaną z ich dalszą rozbudową.
Projekty, oryginalnie zaproponowane, przez firmę HP (takie jak na przykład: \textbf{monasca-api} lub \textbf{monasca-thresh})
są dostosowywane aby zintegrować platformę \textbf{ELKStack} z \textbf{monasca}.
Jeżeli chodzi o część odpowiedzialną za dodania \textbf{LaaS}, autor pracy dyplomowej zajmuje się
tworzeniem nowych rozwiązań, które dotykają więcej niż jednego aspektu. Tematyka rozciąga się
od zadań związanych z \glslink{frontend}{\textbf{warstwą kliencką}} 
do \glslink{backend}{\textbf{warstwy serwera}}, poprzez więcej niż jeden język lub paradygmat programowania.

\newglossaryentry{keystone}{
    name={Keystone},
    description={Aplikacja o kluczowym znaczeniu dla chmury obliczeniowej \textbf{OpenStack}. Dzięki niej możliwe jest tworzenia nowych użytkowników, zarządzania ich uprawnieniami oraz ostatecznie autentykacja i autoryzacja}
}
\newglossaryentry{frontend}{
    name={Frontend},
    description={odnosi się, w przypadku aplikacji komputerowych, do tych ich części, która jest bezpośrednio widoczna dla użytkownika końcowego. Z nią też, użytkownik wchodzi
        w interakcje rozumiane jako akcje. Bezpośrednim ich skutkiem jest komunikacja z warstwą \textbf{backend'u}, przetwarzającą żądania i zwracającą wynik. Rezultaty są później, prezentowane
        użytkownikowi, przykładowo, w postaci tabeli lub okna dialogowego}
}
\newglossaryentry{backend}{
    name={Backend},
    description={w najprostszym tego słowa rozumieniu, znajduje się za częścią
        wizualną aplikacji (\textbf{frontend}) i w oczywisty sposób nie jest bezpośrednio widoczny dla
        użytkownika. Innymi słowy, nie musi wcale wiedzieć o jego istnieniu lub znać jego poziomu złożoności.
        Zadaniem, tej warstwy, jest odebrania żądania, przetworzenie go i zwrócenie wyniku, nie zajmując
        się szczegółami związanymi z graficzną wizualizacją rezultatów}
}
\newglossaryentry{node_js}{
    name={NodeJS},
    description={oparty o silnik \textbf{Chrome V8}, oferuje rozwiązanie pozwalająca na uruchamiania kodu \textbf{JavaScript} poza jej tradycyjnym
        środowiskiem, a mianowicie, jako pełnoprawny program, podobny między innymi do \textbf{Google Chrome} czy też \textbf{Notatnika}. Jego
        dewizą jest \textbf{JavaScript on Server}. Dzięki niemu możliwe jest tworzenia kodu serwera w języku originalnie zaprojektowanym
        do wzbogania interfejsu użytkownika na stronach internetowych, podobnie jak to jest czynione w językach takich jak \textbf{Java} lub \textbf{Python}}
}
\newglossaryentry{wsgi}{
    name={WSGI},
    description={Specyfikacja dla języka Python opisująca komunikacją między serwerem a aplikacją. 
    Stanowi zbiór wytycznych oraz reguł realizujące podwaliny spójnej bazy dla tworzonych programów.
    Dzięki temu większość aplikacji, napisanych dla WSGI, jest w stanie współpracować, bez żadnych lub
    z małymi modyfikacjami, z większością implementacji WSGI
    }
}

\section{Wkład w część praktyczną}
\label{chapter:application:own_work}

Autor, poniższej pracy dyplomowej, zaangażowany był w tworzenia modyfikacji dla praktycznie wszystkich z omawianych powyżej komponentów i w przyszłości będzie kontynuował pracę związaną z ich dalszą rozbudową.
Projekty, oryginalnie zaproponowane, przez firmę HP (takie jak na przykład: \textbf{monasca-api} lub \textbf{monasca-thresh})
są dostosowywane aby zintegrować platformę \textbf{ELKStack} z \textbf{monasca}.
Jeżeli chodzi o część odpowiedzialną za dodania \textbf{LaaS}, autor pracy dyplomowej zajmuje się
tworzeniem nowych rozwiązań, które dotykają więcej niż jednego aspektu. Tematyka rozciąga się
od zadań związanych z \glslink{frontend}{\textbf{frontend'em}} 
do \glslink{backend}{\textbf{backend'u}}, poprzez więcej niż jeden język lub paradygmat programowania.

\subsection{Kibana + Keystone}
\label{chapter:application:own_work:kibana_and_keystone}

\textbf{Kibana} domyślnie nie posiada włączonego żadnego mechanizmu autentykacji użytkownika, który
chciałby przeglądać dane, przechowywane w \textbf{ElasticSearch}(\ref{chapter:application:elkstack:elasticsearch}), 
przy jej użyciu. Istnieje prosty mechanizm, który opiera się na standardowym logowaniu przy użyciu
hasła i nazwy użytkownika. Jednakże, z uwagi na późniejsze plany związane z omawianą aplikacją, a które 
omówione zostaną w kolejnym rozdziale, było to rozwiązanie niewystarczające. Fakt ten wynika z natury
integracji, która odbywa się w chmurze obliczeniowej. Konieczne było rozszerzenie Kibany o możliwość
komunikacji z \glslink{keystone}{\textbf{Keystone}}.

Ponieważ dostęp do \textbf{Kibana}, w przypadku tworzonego projektu, dostępny jest z poziomu \glslink{horizon}{\textbf{Horizon}}, inni
członkowie zespołu zajęli się przygotowaniem logiki odpowiedniej dla uruchomienie jej ze wspomnianego miejsca. Autor pracy 
dyplomowej, zaprojektował odpowiedni mechanizm autentykacji zrealizowany według następujących punktów.

    \subsubsection{Na poziomie serwera}
    W przypadku \textbf{Kibana} należy rozróżnić dwie części składowe - klienta oraz serwer. Dodanie autentykacji
    na poziomie frontend'u nie byłoby rozwiązaniem poprawnym. Użytkownik końcowy miałby możliwość faktycznego
    uruchomienie strony oraz pobrania danych zanim wykryte by zostało, że nie ma koniecznych do tego uprawnień.
    Wolumin logów, a tym samych danych, które \textbf{Kibana} pobiera z \textbf{ElasticSearch} jest bardzo duży.
    Z tego powodu obciążania komponentów aby zwrócić dane potencjalnie niepotrzebnie zostało
    natychmiastowo wykluczone. Z tego też powodu odpowiednia implementacja znalazła się po stronie serwera.
    
    \subsubsection{Jako filtr}
    \begin{listing}
        \begin{minted}{js}
app.use('/elasticsearch', require('./lib/keystone'));
        \end{minted}
        \label{chapter:application:own_work:kibana_and_keystone:filter_code}
        \caption[Autoryzacja z Keystone w Kibana]{Autoryzacja z Keystone w Kibana, źródło: \url{https://github.com/FujitsuEnablingSoftwareTechnologyGmbH/kibana/blob/master/src/server/app.js}}
    \end{listing}
    Zrealizowana została jako filtr. Jest to specyficzna część aplikacji internetowych, która
    uruchamia się dla wszystkich lub wybranych żądań odbieranych przez serwer.
    
    \begin{listing}
        \begin{minted}{js}
var config = require('../../config');
var express = require('express');

var router = module.exports = express.Router();

if (config.keystone) {
    router.use(require('./auth'));
    router.use(function (err, req, res, next) {
        res.status(err.status || 500);
        res.send({message: err.message});
    });
} else {
    router.use(function (req, res, next) {
        next();
    });
}
        \end{minted}
        \label{chapter:application:own_work:kibana_and_keystone:filter_configuration}
        \caption[Autoryzacja z Keystone w Kibana - konfiguracja]{Autoryzacja z Keystone w Kibana - konfiguracja, źródło: \url{https://github.com/FujitsuEnablingSoftwareTechnologyGmbH/kibana/blob/master/src/server/lib/keystone/index.js}}
    \end{listing}
    
    Warto dodać, że pomimo wbudowania filtru do \textbf{Kibana}, nie oznacza to, że musi on być aktywny. Kod
    przedstawiony na listingu \ref{chapter:application:own_work:kibana_and_keystone:filter_configuration} oznacza,
    że będzie on włączony jedynie jeśli w pliku konfiguracyjnym dla serwera \textbf{Kibana} znajdzie się odpowiedni
    segment zawierający następujące informacje:
    \begin{itemize}
        \item adres URL serwera \textbf{Keystone},
        \item port dostępu nieuprzywilejowanego,
        \item port dostępu uprzywilejowanego
    \end{itemize}
    
    Filtr posiada jedno wejście i co najmniej dwa, istotne z punktu widzenia omawianego zagadnienia, wyjścia:
    \begin{itemize}
        \item[sukces] - jeśli autoryzacja użytkownika się powiedzie, filtr zwraca logiczną wartość - prawda,
        która jest sygnałem dla dalszego przetwarzania żądania,
        \item[blokada] - sytuacja odwrotna do sukcesu. W tym momencie, żądania zostaje odrzucone
        i pojawia się odpowiedni komunikat informujący użytkownika o zbyt niskich uprawnieniach.
    \end{itemize}
    
    \subsubsection{Dla komunikacji z ElasticSearch}
    Część serwera \textbf{Kibana} nie zajmuje się jedynie dostarczeniem dostępu do bazy danych
    \textbf{ElasticSearch}. Poza tym, przy jego pomocy, serwowane są media właściwe dla stron internetowych
    (pliki HTML, CSS, JS) lub inne media. Sprawdzania praw użytkowników dla
    każdego z takiego typu żądań okazuje się być wysoce nieefektywne, ponieważ nie ma faktycznego
    przełożenia na aspekt bezpieczeństwa. Z tego tez powodu, jedynie żądania o dane, są wcześniej 
    weryfikowane, aby zezwolić lub zabronić \textbf{Kibana}, i w dalszej kolejności użytkowniki,
    dostępu do nich. Omówione powody, wynikają z natury filtrów, które uruchamiania są dla 
    wszystkich żądań trafiających pod konkretny adres.

Kompletny kod źródłowy aplikacji dostępny jest pod adresem \url{https://github.com/FujitsuEnablingSoftwareTechnologyGmbH/kibana}.

\subsection{keystone-v3-client}
Omówiona w \ref{chapter:application:own_work:kibana_and_keystone} funkcjonalność, zaimplementowana w \textbf{Kibana}, nie byłaby możliwa
bez biblioteki do komunikacji z \textbf{Keystone} w aplikacji napisanej w \glslink{node_js}{\textbf{NodeJS}}.
W momencie jej implementacji, żadna z istniejących bibliotek nie wspierała nowej wersji \textbf{Keystone} - \textbf{V3}. 
Było to bodźcem dla napisania własnej, której autor poniższej pracy dyplomowej, jest, na obecną chwilę, jedynym autorem. Zwrot \textit{obecna chwila}
jest tutaj istotny, ponieważ \textbf{biblioteka} została udostępniona publicznie na licencji \textbf{Apache V2}. Skutkiem tego jest to, że
każdy programista z dowolnego miejsca na świecie, może stać się jej współtwórcą. 

Jak zostało wspomniane powyżej, nadrzędnym celem \textbf{keystone-v3-client} jest dostarczenie programistom \textbf{NodeJS} możliwości
komunikacji z nową wersją \textbf{Keystone}. Sam \textbf{Keystone} jest tak naprawdę uruchamiany jako serwer, z którym komunikować się można,
wysyłając żądania HTTP. Lista możliwych żądań jest zbyt długa, aby w pełni ją tutaj przedstawić 
\footnote{Pełna specyfikacja \textbf{Keystone} dostępna jest pod adresem \url{http://developer.openstack.org/api-ref-identity-v3.html}}. Jednak,
dla każdego z możliwych adresów \textbf{keystone-v3-client} udostępnia odpowiednie funkcje, która pozwalają te żądania wykonać.
    \begin{listing}
        \begin{minted}{js}
module.exports.tokens = require('./tokens');
module.exports.service_catalog = require('./service-catalog');
module.exports.endpoints = require('./endpoints');
module.exports.domains = require('./domains');
module.exports.projects = require('./projects');
module.exports.users = require('./users');
module.exports.groups = require('./groups');
module.exports.credentials = require('./credentials');
module.exports.roles = require('./roles');
module.exports.policies = require('./policies');
        \end{minted}
        \label{chapter:application:own_work:keystone_v3_client:endpoints}
        \caption[Definicja obsługiwanych adresów \textbf{Keystone}]{
            Definicja obsługiwanych adresów \textbf{Keystone} jako moduł NodeJS, źródło: \url{https://github.com/FujitsuEnablingSoftwareTechnologyGmbH/keystone-v3-client/blob/master/lib/keystone/index.js}}
    \end{listing}

Kompletny kod źródłowy biblioteki dostępny jest pod adresem \url{https://github.com/FujitsuEnablingSoftwareTechnologyGmbH/keystone-v3-client}.
Została ona również opublikowana na \url{https://www.npmjs.com}. 
Opis paczki znaleźć można pod adresem \url{https://www.npmjs.com/package/keystone-v3-client}

\subsection{monasca-log-api}

    \subsubsection{monasca-log-api - Java}
    Implementacja \textbf{monasca-log-api} pojawiła się jako pierwsza. Była ona suplementem dla \textbf{monasca-api}, również napisanego w Java.
    Jednakże, podczas gdy drugi z omawianych komponentów służy do pracy z alarmami oraz metrykami, \textbf{monasca-log-api} pozwala na zapisywania
    logów do bazy danych. Autor pracy dyplomowej, mimo że nie jest oryginalnym autorem tego rozwiązania, miał duży wkład w następujące aspekty
    implementacji:
    \begin{itemize}
        \item poprawa logiki aplikacji odnośnie dodawania meta danych do odbieranych logów,
        \item naprawa zauważonych błędów,
        \item podniesienie jakości kodu,
        \item lepszą walidacją rozmiaru żądania, tym samym maksymalnego rozmiaru logu
    \end{itemize}

Odpowiedź na trend w przenoszeniu komponentów, będących częścią chmury obliczeniowej OpenStack, na język Python. Autor, poniższej pracy dyplomowej, jest autorem tej implementacji \textbf{monasca-log-api}. Na obecną chwilę jest ona komplementarna do wersji w języku Java. Całość została oparta o szybki serwer \textbf{Gunicorn} kompatybilny z \glslink{wsgi}{\textbf{WSGI}}. Dzięki
temu aplikacja może zostać łatwo przeniesiona na inną implementacją \textbf{WSGI}, bez większych trudności,
jeśli zajdzie taka potrzeba. Implementacja obejmuje:
\begin{itemize}
    \item przygotowanie całej architektury aplikacji, włącznie z:
    \begin{itemize}
        \item ładowaniem ustawień,
        \item rozłożeniem kodu na logiczne, współpracuje ze sobą, elementy,
        \item klasy odpowiedzialne za obsługę żądań HTTP
    \end{itemize}
    \item utworzeniem logiki odnośnie walidacji nadchodzących wiadomości:
    \begin{itemize}
        \item poprawna struktura nagłówków żądania,
        \item dwu etapowa walidacja rozmiaru wiadomości,
    \end{itemize}
    \item autoryzacja dostępu,
    \item mechanizm weryfikacji stanu (z angielskiego \textbf{healtcheck}),
    \item tworzenia struktury gotowej do przekazania dla \textbf{monasca-log-transformer}
\end{itemize}

Warto w tym miejscu dodać, że \textbf{monasca-log-api}, w języku Python, spotkała się z dużo
większym odzewem ze strony społeczności. Na obecną chwilę, kod aplikacji, ma więcej niż 1 kontrybutora.
Kod, wersji Java oraz Python, znajduje się pod adresem \url{https://github.com/openstack/monasca-log-api/}.
Dodatkowo, równoległe do samego API, autor pracy dyplomowej, rozszerzył rolę \textbf{Ansible} (\ref{chapter:application:architecture:installation:ansible}) o
możliwość instalacji, na wskazanych hostach, napisanej przez siebie implementacji. Kod źródłowy, wspomnianego komponentu, znajduje się
pod adresem \url{https://github.com/FujitsuEnablingSoftwareTechnologyGmbH/ansible-monasca-log-api}.

\subsection{ORM dla monasca-api oraz monasca-thresh}

\subsection{Problem wielu linii w monasca-log-agent}

\subsection{Rotacja logów}

\subsection{Ocena kodu tworzonego w społeczności}

\subsection{Nominacja do status code-reviewer}

\todo[inline]{przejrzeć resztę kontrybucji i wpisać}