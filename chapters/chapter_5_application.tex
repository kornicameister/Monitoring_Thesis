\chapter{Część praktyczna - wkład własny}
\label{chapter:application_own}

\newglossaryentry{keystone}{
    name={Keystone},
    description={Aplikacja o kluczowym znaczeniu dla chmury obliczeniowej \textbf{OpenStack}. Dzięki niej możliwe jest tworzenia nowych użytkowników, zarządzania ich uprawnieniami oraz ostatecznie autentykacja i autoryzacja}
}
\newglossaryentry{frontend}{
    name={Frontend},
    description={odnosi się, w przypadku aplikacji komputerowych, do tych ich części, która jest bezpośrednio widoczna dla użytkownika końcowego. Z nią też, użytkownik wchodzi
        w interakcje rozumiane jako akcje. Bezpośrednim ich skutkiem jest komunikacja z warstwą \textbf{backend'u}, przetwarzającą żądania i zwracającą wynik. Rezultaty są później, prezentowane
        użytkownikowi, przykładowo, w postaci tabeli lub okna dialogowego}
}
\newglossaryentry{backend}{
    name={Backend},
    description={w najprostszym tego słowa rozumieniu, znajduje się za częścią
        wizualną aplikacji (\textbf{frontend}) i w oczywisty sposób nie jest bezpośrednio widoczny dla
        użytkownika. Innymi słowy, nie musi wcale wiedzieć o jego istnieniu lub znać jego poziomu złożoności.
        Zadaniem, tej warstwy, jest odebrania żądania, przetworzenie go i zwrócenie wyniku, nie zajmując
        się szczegółami związanymi z graficzną wizualizacją rezultatów}
}
\newglossaryentry{node_js}{
    name={NodeJS},
    description={oparty o silnik \textbf{Chrome V8}, oferuje rozwiązanie pozwalająca na uruchamiania kodu \textbf{JavaScript} poza jej tradycyjnym
        środowiskiem, a mianowicie, jako pełnoprawny program, podobny między innymi do \textbf{Google Chrome} czy też \textbf{Notatnika}. Jego
        dewizą jest \textbf{JavaScript on Server}. Dzięki niemu możliwe jest tworzenia kodu serwera w języku originalnie zaprojektowanym
        do wzbogania interfejsu użytkownika na stronach internetowych, podobnie jak to jest czynione w językach takich jak \textbf{Java} lub \textbf{Python}}
}
\newglossaryentry{wsgi}{
    name={WSGI},
    description={Specyfikacja dla języka Python opisująca komunikacją między serwerem a aplikacją. 
        Stanowi zbiór wytycznych oraz reguł realizujące podwaliny spójnej bazy dla tworzonych programów.
        Dzięki temu większość aplikacji, napisanych dla WSGI, jest w stanie współpracować, bez żadnych lub
        z małymi modyfikacjami, z większością implementacji WSGI
    }
}
\newglossaryentry{orm}{
    name={ORM},
    description={
        Mapowanie obiektowo-relacyjne jest techniką według, której możliwe jest reprezentowanie bazy danych pod postacią obiektów.
        Inną nadrzędną cechą jest możliwość współpracy z więcej niż jednym silnikiem bazodanowym bez konieczności,
        lub z małymi zmianami, bazy kodu.
    }
}

\section{Wkład w część praktyczną}
\label{chapter:application_own:own_work}

Autor, poniższej pracy dyplomowej, zaangażowany był w tworzenia modyfikacji dla praktycznie wszystkich z omawianych powyżej komponentów i w przyszłości będzie kontynuował pracę związaną z ich dalszą rozbudową.
Projekty, oryginalnie zaproponowane, przez firmę HP (takie jak na przykład: \textbf{monasca-api} lub \textbf{monasca-thresh})
są dostosowywane aby zintegrować platformę \textbf{ELKStack} z \textbf{monasca}.
Jeżeli chodzi o część odpowiedzialną za dodania \textbf{LaaS}, autor pracy dyplomowej zajmuje się
tworzeniem nowych rozwiązań, które dotykają więcej niż jednego aspektu. Tematyka rozciąga się
od zadań związanych z \glslink{frontend}{\textbf{frontend'em}} 
do \glslink{backend}{\textbf{backend'u}}, poprzez więcej niż jeden język lub paradygmat programowania.

\section{ansible-monasca-kibana}
\label{chapter:application:own_work:ansible_kibana}

    Rola Ansible (\ref{chapter:application:architecture:installation:ansible}) służąca do 
    instalacji klienta stosu \textbf{ELKStack} (\ref{chapter:application:elkstack}), czyli programu Kibana.
    Oprócz wspomnianej instalacji, rola umożliwia również:
    \begin{itemize}
        \item zmianę konfiguracji programu bez konieczności jego ponownej instalacji,
        \item weryfikacją stanu poprzez sprawdzenia zajętości portu TCP,
        \item startowanie oraz zatrzymywanie działania aplikacji
    \end{itemize}
    Dodatkowo rola wspiera mechanizmy izolacji procesów dla platformy Unix. W ramach przygotowania
    do właściwiej instalacji tworzony jest użytkownik oraz grupa. Proces aplikacji uruchamiany będzie później
    pod ich kontrolą, a sam program \textbf{Kibana} posiadał będzie dostęp jedynie do tych zasobów
    systemowych, których właścicielem jest, wspomniany powyżej, użytkownik będący członkiem konkretnej grupy.
    Jest to szczególnie istotne zabezpieczenie, ponieważ zapobiega ona nieautoryzowany dostępowi aplikacji,
    do plików lub innych zasobów systemowych, których nie powinna ona modyfikować.
    
    Całość implementacji przypada autorowi pracy dyplomowej. Kod źródłowy powyższego komponentu jest
    dostępny pod adresem \\ \burl{https://github.com/FujitsuEnablingSoftwareTechnologyGmbH/ansible-monasca-kibana}.

\section{ansible-monasca-elasticsearch}
\label{chapter:application:own_work:ansible_elasticsearch}

    Podobnie jak \textbf{ansible-monasca-kibana} (\ref{chapter:application:own_work:ansible_kibana}), opisywana
    rola służy instalacji jednej z części składowych stosu \textbf{ELKStack} (\ref{chapter:application:elkstack}) - 
    serwera bazodanowego \textbf{ElasticSearch}. Możliwości roli są identyczne do tych opisanych dla roli 
    \textbf{ansible-monasca-kibana}.
    
    Implementacja całej roli jest dziełem autora pracy dyplomowej, a jej kod źródłowy dostępny
    jest pod adresem \burl{https://github.com/FujitsuEnablingSoftwareTechnologyGmbH/ansible-monasca-elasticsearch}.

\section{ansible-monasca-log-transformer}
\label{chapter:application:own_work:ansible_log_transformer}

    W odróżnieniu od \textbf{ansible-monasca-kibana} oraz \textbf{ansible-monasca-elasticsearch}, zadaniem roli
    \textbf{ansible-monasca-log-transformer} jest instalacja na wybranym serwerze, lub serwerach, programu
    \textbf{Logstash} wraz ze specjalną konfiguracją opisaną w rozdziale \ref{chapter:monasca:monasca_log_transformer}.
    
    Dzięki roli możliwe jest:
    \begin{itemize}
        \item zainstalowanie komponentu,
        \item startowanie oraz zatrzymywanie jego działania,
        \item zmiana konfiguracji
    \end{itemize}
    
    Niestety, z uwagi na specyfikę instalowanego programu, którym jest Logstash, nie jest możliwa weryfikacja stanu
    działania aplikacji.
    
    Wstępna implementacja roli jest autorstwa poniższej pracy dyplomowej i może zostać oszacowana na 90\% całego
    kodu źródłowego, który dostępny jest pod adresem
    \burl{https://github.com/FujitsuEnablingSoftwareTechnologyGmbH/ansible-monasca-log-transformer}.

\section{ansible-monasca-log-persister}
\label{chapter:application:own_work:ansible_log_persister}

    Zadaniem omawianej roli jest instalacja programu \textbf{Logstash} wraz z konfiguracją opisaną
    w rozdziale \ref{chapter:monasca:monasca_log_persister}. Funkcjonalność oferowane przez rolę jest
    zbliżone do tej opisanej dla \textbf{ansible-monaca-log-transformer} z jedną szczególną różnicą.
    Możliwe jest zweryfikowanie, czy na wybranym serwerze aplikacja jest działa. Całość weryfikacja oparta jest
    na specyfice komponentu \textbf{monasca-log-persister}. Dane odczytane z kolejki \textbf{Kafka}, po oczyszczeniu
    wysyłane są do serwera \textbf{ElasticSearch}. Aby ta operacja była możliwa, \textbf{Logstash} otwiera konkretny
    port TCP. Zajętość tego portu jest sygnałem, że \textbf{monasca-log-persister} został poprawnie uruchomiony i wysyła
    dane do \textbf{ElasticSearch}.

Całość implementacji przypada autorowi pracy dyplomowej i jest dostępna pod adresem
\burl{https://github.com/FujitsuEnablingSoftwareTechnologyGmbH/ansible-monasca-log-persister}.

\section{ansible-monasca-log-api}
\label{chapter:application:own_work:ansible_monasca_log_api}

    Autor, poniższej pracy dyplomowej, odpowiedzialny był w przypadku omawianej roli głównie za implementację
    logiki, która pozwala by na zainstalowanie \textbf{monasca-log-api} (\ref{chapter:monasca:monasca_log_api}),
    a konkretnie implementacji napisanej w języku Python (\ref{chapter:application:own_work:monasca_log_api:python}).
    
    Rozszerzenie istniejącej roli sprowadziło się do:
    \begin{itemize}
        \item ogólnej poprawy struktury roli poprzez wyodrębnienie odrębnych bloków funkcjonalnych, tworząc
        oddzielne pliki zawierającej ich kod, pozwalających na:
        \begin{itemize}
            \item przygotowanie środowiska do instalacji aplikacji,
            \item właściwego zainstalowania programu na wybranych serwerach lub serwerze,
            \item konfiguracji,
            \item startowania oraz zatrzymywania procesu aplikacji,
            \item weryfikowania stanu
        \end{itemize}
        \item zachowaniu funkcjonalności starej roli, pozwalając w dalszym ciągu na instalację implementacji
        napisanej w języku Java,
        \item dodaniu możliwości instalacji oraz konfiguracji dla implementacji w języku Python
    \end{itemize}
    
    Wkład, twórcy poniższej pracy dyplomowej, ocenić można na około 60\%, a całość kodu źródłowego
    dostępna jest pod adresem \\ \burl{https://github.com/FujitsuEnablingSoftwareTechnologyGmbH/ansible-monasca-log-api}.

\section{Kibana + Keystone}
\label{chapter:application:own_work:kibana_and_keystone}

\textbf{Kibana} domyślnie nie posiada włączonego żadnego mechanizmu autentykacji użytkownika, który
chciałby przeglądać dane, przechowywane w \textbf{ElasticSearch}(\ref{chapter:application:elkstack:elasticsearch}), 
przy jej użyciu. Istnieje prosty mechanizm, który opiera się na standardowym logowaniu przy użyciu
hasła i nazwy użytkownika. Jednakże, z uwagi na późniejsze plany związane z omawianą aplikacją, a które 
omówione zostaną w kolejnym rozdziale, było to rozwiązanie niewystarczające. Fakt ten wynika z natury
integracji, która odbywa się w chmurze obliczeniowej. Konieczne było rozszerzenie Kibany o możliwość
komunikacji z \glslink{keystone}{\textbf{Keystone}}.

Ponieważ dostęp do \textbf{Kibana}, w przypadku tworzonego projektu, dostępny jest z poziomu \glslink{horizon}{\textbf{Horizon}}, inni
członkowie zespołu zajęli się przygotowaniem logiki odpowiedniej dla uruchomienie jej ze wspomnianego miejsca. Autor pracy 
dyplomowej, zaprojektował odpowiedni mechanizm autentykacji zrealizowany według następujących punktów.

    \subsection{Na poziomie serwera}
    W przypadku \textbf{Kibana} należy rozróżnić dwie części składowe - klient oraz serwer. Dodanie autentykacji
    na poziomie warstwy klienta nie byłoby rozwiązaniem poprawnym. Użytkownik końcowy miałby możliwość faktycznego
    uruchomienie strony oraz pobrania danych zanim wykryte by zostało, że nie ma koniecznych do tego uprawnień.
    Wolumin logów, a tym samych danych, które \textbf{Kibana} pobiera z \textbf{ElasticSearch} jest bardzo duży.
    Z tego powodu obciążania komponentów aby zwrócić dane potencjalnie niepotrzebnie zostało
    natychmiastowo wykluczone. Z tego też powodu odpowiednia implementacja znalazła się po stronie serwera.
    
    \subsection{Jako filtr}
    \begin{listing}
        \begin{minted}{js}
app.use('/elasticsearch', require('./lib/keystone'));
        \end{minted}
        \label{chapter:application_own:own_work:kibana_and_keystone:filter_code}
        \caption[Autoryzacja z Keystone w Kibana]{Autoryzacja z Keystone w Kibana, źródło: \url{https://github.com/FujitsuEnablingSoftwareTechnologyGmbH/kibana/blob/master/src/server/app.js}}
    \end{listing}
    Zrealizowana została jako filtr. Jest to specyficzna część aplikacji internetowych, która
    uruchamia się dla wszystkich lub wybranych żądań odbieranych przez serwer.
    
    \begin{listing}
        \begin{minted}{js}
var config = require('../../config');
var express = require('express');

var router = module.exports = express.Router();

if (config.keystone) {
    router.use(require('./auth'));
    router.use(function (err, req, res, next) {
        res.status(err.status || 500);
        res.send({message: err.message});
    });
} else {
    router.use(function (req, res, next) {
        next();
    });
}
        \end{minted}
        \caption[Autoryzacja z Keystone w Kibana - konfiguracja]{Autoryzacja z Keystone w Kibana - konfiguracja, źródło: \url{https://github.com/FujitsuEnablingSoftwareTechnologyGmbH/kibana/blob/master/src/server/lib/keystone/index.js}}
        \label{chapter:application_own:own_work:kibana_and_keystone:filter_configuration}
    \end{listing}
    
    Warto dodać, że pomimo wbudowania filtru do \textbf{Kibana}, nie oznacza to, że musi on być aktywny. Kod
    przedstawiony na listingu \ref{chapter:application_own:own_work:kibana_and_keystone:filter_configuration} oznacza,
    że będzie on włączony jedynie jeśli w pliku konfiguracyjnym dla serwera \textbf{Kibana} znajdzie się odpowiedni
    segment zawierający następujące informacje:
    \begin{itemize}
        \item adres URL serwera \textbf{Keystone},
        \item port dostępu nieuprzywilejowanego,
        \item port dostępu uprzywilejowanego
    \end{itemize}
    
    Filtr posiada jedno wejście i co najmniej dwa, istotne z punktu widzenia omawianego zagadnienia, wyjścia:
    \begin{itemize}
        \item[sukces] - jeśli autoryzacja użytkownika się powiedzie, filtr zwraca logiczną wartość - prawda,
        która jest sygnałem dla dalszego przetwarzania żądania,
        \item[blokada] - sytuacja odwrotna do sukcesu. W tym momencie, żądania zostaje odrzucone
        i pojawia się odpowiedni komunikat informujący użytkownika o zbyt niskich uprawnieniach.
    \end{itemize}
    
    \subsection{Dla komunikacji z ElasticSearch}
    Część serwera \textbf{Kibana} nie zajmuje się jedynie dostarczeniem dostępu do bazy danych
    \textbf{ElasticSearch}. Poza tym, przy jego pomocy, serwowane są media właściwe dla stron internetowych
    (pliki HTML, CSS, JS) lub inne media. Sprawdzania praw użytkowników dla
    każdego z takiego typu żądań okazuje się być wysoce nieefektywne, ponieważ nie ma faktycznego
    przełożenia na aspekt bezpieczeństwa. Z tego tez powodu, jedynie żądania o dane, są wcześniej 
    weryfikowane, aby zezwolić lub zabronić \textbf{Kibana}, i w dalszej kolejności użytkowniki,
    dostępu do nich. Omówione powody, wynikają z natury filtrów, które uruchamiania są dla 
    wszystkich żądań trafiających pod konkretny adres.

Kompletny kod źródłowy aplikacji dostępny jest pod adresem \url{https://github.com/FujitsuEnablingSoftwareTechnologyGmbH/kibana}.

\section{keystone-v3-client}
Omówiona w \ref{chapter:application:own_work:kibana_and_keystone} funkcjonalność, zaimplementowana w \textbf{Kibana}, nie 
byłaby możliwa bez biblioteki do komunikacji z \textbf{Keystone} w aplikacji napisanej w \glslink{node_js}{\textbf{NodeJS}}.
W momencie jej implementacji, żadna z istniejących bibliotek nie wspierała nowej wersji \textbf{Keystone} - \textbf{V3}. 
Było to bodźcem dla napisania własnej, której autor poniższej pracy dyplomowej, jest, na obecną chwilę, jedynym autorem. Zwrot \textit{obecna chwila}
jest tutaj istotny, ponieważ \textbf{biblioteka} została udostępniona publicznie na licencji \textbf{Apache V2}. Skutkiem tego jest to, że
każdy programista z dowolnego miejsca na świecie, może stać się jej współtwórcą. 

Jak zostało wspomniane powyżej, nadrzędnym celem \textbf{keystone-v3-client} jest dostarczenie programistom \textbf{NodeJS} możliwości
komunikacji z nową wersją \textbf{Keystone}. Sam \textbf{Keystone} jest tak naprawdę uruchamiany jako serwer, z którym komunikować się można,
wysyłając żądania HTTP. Lista możliwych żądań jest zbyt długa, aby w pełni ją tutaj przedstawić 
\footnote{Pełna specyfikacja \textbf{Keystone} dostępna jest pod adresem \url{http://developer.openstack.org/api-ref-identity-v3.html}}. Jednak,
dla każdego z możliwych adresów \textbf{keystone-v3-client} udostępnia odpowiednie funkcje, która pozwalają te żądania wykonać.
    \begin{listing}
        \begin{minted}{js}
module.exports.tokens = require('./tokens');
module.exports.service_catalog = require('./service-catalog');
module.exports.endpoints = require('./endpoints');
module.exports.domains = require('./domains');
module.exports.projects = require('./projects');
module.exports.users = require('./users');
module.exports.groups = require('./groups');
module.exports.credentials = require('./credentials');
module.exports.roles = require('./roles');
module.exports.policies = require('./policies');
        \end{minted}
        \caption[Definicja obsługiwanych adresów \textbf{Keystone}]{
            Definicja obsługiwanych adresów \textbf{Keystone} jako moduł NodeJS, źródło: \url{https://github.com/FujitsuEnablingSoftwareTechnologyGmbH/keystone-v3-client/blob/master/lib/keystone/index.js}}
        \label{chapter:application_own:own_work:keystone_v3_client:endpoints}
    \end{listing}

Kompletny kod źródłowy biblioteki dostępny jest pod adresem \url{https://github.com/FujitsuEnablingSoftwareTechnologyGmbH/keystone-v3-client}.
Została ona również opublikowana na \url{https://www.npmjs.com}. 
Opis paczki znaleźć można pod adresem \url{https://www.npmjs.com/package/keystone-v3-client}

\section{monasca-log-api}

\subsection{monasca-log-api - Java}
\label{chapter:application:own_work:monasca_log_api:java}
    Implementacja \textbf{monasca-log-api} pojawiła się jako pierwsza. Była ona suplementem dla \textbf{monasca-api}, również 
    napisanego w Java. Jednakże, podczas gdy drugi z omawianych komponentów służy do pracy z alarmami oraz metrykami, 
    \textbf{monasca-log-api} pozwala na zapisywania logów do bazy danych. Autor pracy dyplomowej, mimo że nie jest 
    oryginalnym autorem tego rozwiązania, miał duży wkład w następujące aspekty implementacji:
    \begin{itemize}
        \item poprawa logiki aplikacji odnośnie dodawania meta danych do odbieranych logów,
        \item naprawa zauważonych błędów,
        \item podniesienie jakości kodu,
        \item lepszą walidacją rozmiaru żądania, tym samym maksymalnego rozmiaru logu
    \end{itemize}
    
    Wkład, autora pracy dyplomowej, dla \textbf{monasca-log-api} w języku Java, można oszacować na 20\%.

\subsection{monasca-log-api - Python}
\label{chapter:application:own_work:monasca_log_api:python}

Odpowiedź na trend w przenoszeniu komponentów, będących częścią chmury obliczeniowej OpenStack, na język Python. Autor, poniższej pracy dyplomowej, jest autorem tej implementacji \textbf{monasca-log-api}. Na obecną chwilę jest ona komplementarna do wersji w języku Java. Całość została oparta o szybki serwer \textbf{Gunicorn} kompatybilny z \glslink{wsgi}{\textbf{WSGI}}. Dzięki
temu aplikacja może zostać łatwo przeniesiona na inną implementacją \textbf{WSGI}, bez większych trudności,
jeśli zajdzie taka potrzeba. Implementacja obejmuje:
\begin{itemize}
    \item przygotowanie całej architektury aplikacji, włącznie z:
    \begin{itemize}
        \item ładowaniem ustawień,
        \item rozłożeniem kodu na logiczne, współpracuje ze sobą, elementy,
        \item klasy odpowiedzialne za obsługę żądań HTTP
    \end{itemize}
    \item utworzeniem logiki odnośnie walidacji nadchodzących wiadomości:
    \begin{itemize}
        \item poprawna struktura nagłówków żądania,
        \item dwu etapowa walidacja rozmiaru wiadomości,
    \end{itemize}
    \item autoryzacja dostępu,
    \item mechanizm weryfikacji stanu (z angielskiego \textbf{healtcheck}),
    \item tworzenia struktury gotowej do przekazania dla \textbf{monasca-log-transformer}
\end{itemize}

Warto w tym miejscu dodać, że \textbf{monasca-log-api}, w języku Python, spotkała się z dużo
większym odzewem ze strony społeczności. Na obecną chwilę, kod aplikacji, ma więcej niż 1 kontrybutora.
Kod, wersji Java oraz Python, znajduje się pod adresem \url{https://github.com/openstack/monasca-log-api/}.
Równolegle do przygotowania implementacji w języku Python, autor pracy dyplomowej, rozszerzył rolę
\textbf{ansible-monasca-log-api} opisaną w rozdziale \ref{chapter:application:own_work:ansible_monasca_log_api}.
Wkład, twórcy pracy magisterskiej, można oszacować na poziomie 95\% dla \textbf{monasca-log-api} w wersji
Python.

\section{ORM dla monasca-api oraz monasca-thresh}
\label{chapter:application:own_work:orm}

Oryginalnie oba z komponentów zostały zaprojektowane z myślą o wsparciu dla relacyjnej bazy danych \textbf{MySQL}. 
Pierwszą z kontrybucji dla \textbf{monasca} było rozszerzenie obsługi baz danych o \textbf{PostgreSQL}. Nie zostało
to jednak osiągnięte poprzez dodania nowego zestawu klas, napisanych z myślą o wsparciu jedynie tej bazy danych. Zamiast
tego, wprowadzono rozwiązanie znane pod nazwą \glslink{frontend}{textbf{ORM}}.
Autor, poniższej pracy dyplomowej, nie jest oryginalnym autorem rozwiązania. Propozycja, która zostało,
przedstawiona społeczności, jest dziełem jednego z programistów projektu. Niemniej, twórca pracy magisterskiej był, w późniejszym czasie,
odpowiedzialny za następujące aspekty tej części całego projektu:

    \subsection{Poprawa jakości i struktury kodu}
    W ramach tego punktu, autor, zajął się między innymi ogólnym podniesienie jakości. Zrealizowane to zostało
    na trzech płaszczyznach:
    \begin{itemize}
        \item[monasca-common]
        Przepisania od nowa modelu mapowania obiektowo-relacyjnego. Warto dodać, że projekt \textbf{monasca-common} stanowi
        wspólną bazę dla wszystkich projektów \textbf{monasca}. Z tego też powodu, model danych zaimplementowany jest
        właśnie w tym projekcie. Nowa wersja warstwy modelu danych cechuje się:
        \begin{itemize}
            \item wspólną bazą dla kluczy głównych, które zdefiniowane są jedynie na poziomie klasy abstrakcyjnej,
            \item jednolitym podejściem do zarządzania obiektami opisujących dane czasowe,
            \item zgodnym z zalecaniami sposobem obliczenia wartości \textit{hashCode} oraz \textit{equals}
            \footnote{\textit{hashCode} oraz \textit{equals} - dwie podstawowe metody implementowane przez każdą klasę w języku Java. Stanowią one
                kluczowy element, między innymi, w pracy z kolekcjami}
        \end{itemize}
        \item[monasca-api, monasca-thresh]
        Dopasowania repozytoriów danych do nowego modelu obiektowego. Największym sukcesem było
        wykorzystanie wstępnie przygotowanych zapytań HQL\footnote{HQL - Hibernate Query Language, jest to
            pseudo obiektowy język zapytań, podobny do SQL}, zamiast tych, które wykorzystywały łączenie
        łańcuchów znakowych. Takie podejście, jest mało eleganckie i trudne w utrzymaniu. Zaletą HQL są
        parametry posiadające własną nazwę. Dodatkowo, tam gdzie było to konieczne i możliwe, autor
        zastosował dynamicznie budowane zapytania, oparte o 
        \href{https://docs.jboss.org/hibernate/orm/5.0/devguide/en-US/html_single/#d5e3489}{Criteria API}.
        Dla potencjalnie dużych zbiorów danych, autor zastosował mechanizm podobny do kursora, aby
        dostarczyć wymaganej funkcjonalności. Ostatecznie był on także autorem poprawek do testów jednostkowych,
        bazujących na tych, które oryginalnie stworzone zostały na potrzeby repozytoriów, komunikujących się
        z \textbf{MySQL}.
    \end{itemize}
    
    \subsection{Dopasowanie rozwiązania do MySQL i PostgreSQL}
    Mimo, że zastosowanym rozwiązaniem był ORM, okazało się, że dopasowanie modelu danych, tak by pasował do 
    schematu bazy danych \textbf{MySQL} oraz \textbf{PostgreSQL}, jest obarczone dużym stopniem trudności. 
    W dużej mierze napotkane przeszkody wynikały z:
    \begin{itemize}
        \item zastosowaniu dla \textbf{MySQL} typów danych SQL właściwych jedynie dla tej bazy danych,
        \item trudną w zamodelowaniu specyficznej struktury oraz zależności między tabelami,
        \item oparciem relacji dla niektórych encji o typ danych UUID
    \end{itemize}
    
    Warto nadmienić, że część ze wspomnianych problemów została zaadresowane również
    w projekcie \href{https://github.com/FujitsuEnablingSoftwareTechnologyGmbH/ansible-monasca-schema}{ansible-monasca-schema}.
    Autor poniższej pracy dyplomowej, był odpowiedzialny za:
    \begin{itemize}
        \item zamianę typu danych \textit{ENUM} na model reprezentacji tego typu wartości, oparty
        o pojedyncza tabelę. Głównymi zaletami tego rozwiązania są:
        \begin{itemize}
            \item zachowanie wymuszania spójności danych,
            \item możliwość modyfikacji stałych poprzez operacji \textbf{INSERT}, \textbf{DELETE} oraz \textbf{UPDATE}. Warto
            dodać, że dla tabeli tego typu zmiany są operacją trywialną. Nie jest to jednak prawda dla wartości ENUM, ponieważ
            wymaga to zmian bezpośredniości na deklaracji DDL danej tabeli
        \end{itemize}
        Warto dodać, że konieczne było zaprojektowanie nowych tabel, aby posiadały jedną tylko kolumnę, będącą jednocześnie
        wartością oraz kluczem głównym. Celowe złamania zasady normalizacji baz danych podyktowane było, koniecznością
        dopasowania modelu danych, aby pasował zarówno do \textbf{MySQL} oraz \textbf{PostgreSQL} oraz, co ważniejsze,
        zminimalizowaniem ilości koniecznych do wprowadzania zmian w \textbf{monasca-api}, na poziomie warstwy dostępu do danych.
        \item zamianą typu danych \textbf{MEDIUMTEXT} na \textbf{LONGTEXT}. Podczas gdy drugi ze wspomnianych typów jest równoważnie
        wspierany dla obu baz danych, \textbf{MEDIUMTEXT} działa prawidłowo jedynie dla \textbf{MySQL}
    \end{itemize}
    
    Ostatecznie, oprócz zmian na poziomie modelu danych oraz schematu bazy danych, autor miał wkład w poprawę oraz udoskonalenie
    kodu warstwy dostępu do danych. Najważniejszymi aspektami tej części było utworzenia odpowiednich klas, implementujących logikę
    biznesową, oryginalnie dostępną jedynie dla \textbf{MySQL}. Podstawowym warunkiem akceptacji ze strony społeczności było
    odzwierciedlania wszystkich założeń zaimplementowanych dla \textbf{MySQL}, tak aby możliwe było wykorzystanie mapowania obiektowo-relacyjnego
    zarówno z bazą danych \textbf{PostgreSQL}, jak i \textbf{MySQL}.
    
    \subsection{Migracja bazy danych w Ansible}
    \label{chapter:application:own_work:orm:ansible_migration}
    Dopełnieniem wprowadzanie \textbf{ORM} dla projektów \textbf{monasca-api} oraz \textbf{monasca-thresh} było spełnienie
    wymagań społeczności odnośnie pakietu migracji danych. Żądania dodania tego typu funkcjonalności było następstwem zmian, jakie
    zostały wprowadzona na poziomie definicji schematu bazy danych dla \textbf{MySQL}. W ramach implementacji, autor poniższej pracy dyplomowej,
    rozszerzył rolę \textbf{ansible-monasca-schema} o skrypt migracyjny, który miał zapewnić:
    \begin{itemize}
        \item zamianę wartości typu \textbf{ENUM} na łańcuchy tekstowe oraz utworzenie dodatkowych tabel
        \item napisanie procedury, której zadaniem było konwersja danych na nowy format przechowywania,
        \item zmiana typów kolumn, które były kompatybilne jedynie z \textbf{MySQL}
    \end{itemize}
    
    \subsection{Oszacowany wkład}
    Z uwagi na duży rozmiar obu wspomnianych projektów oraz liczbę kontrybutorów, trudno jest oszacować procentowy wkład 
    autora pracy dyplomowej. Dodatkowo takie oszacowanie komplikuje fakt, że wprowadzenie warstwy \textbf{ORM} dla 
    \textbf{monasca-api} oraz \textbf{monasca-thresh}, nie są jedynymi kontrybucjami twórcy pracy magisterskiej. Niemniej, 
    łączny wkład autora, wyrażony w postaci metryki \textbf{LOC - Lines of code} przedstawia się następująco:
    \begin{itemize}
        \item dla \textbf{monasca-api} jest to 3890 zmienionych linii kodu,
        \item dla \textbf{monasca-thresh}, ta sama metryka przyjmuje wartość 1589 linii.
    \end{itemize}
    Większość z nich przypada na implementację mapowania obiektowo-relacyjnego.
    Dla roli \textbf{ansible-monasca-schema}, ilość zmienionych linii kodu wynosi 1459. Całość kontrybucji odnosi się do 
    zmian opisanych w rozdziale \ref{chapter:application:own_work:orm:ansible_migration}.

\section{Problem wielu linii w monasca-log-agent}
\label{chapter:application_own:own_work:monasca_log_agent}

    \subsection{Przedstawienie problemu}
    \begin{listing}[h]
        \begin{minted}{yaml}
        {
            name: "multiline",
            type: '"go"',
            configuration: {
                pattern: '"^\[(?!signal)\w+\]"',
                what: '"previous"',
                negate: '"true"'
            }
        }
        \end{minted}
        \caption[Detekcja wielu liniach dla języka GOLang]{
            Detekcja wielu liniach dla języka GOLang, źródło: \url{https://github.com/FujitsuEnablingSoftwareTechnologyGmbH/ansible-monasca-log-agent/blob/master/vars/main.yml}}
        \label{chapter:application_own:own_work:monasca_log_agent:filter_example}
    \end{listing}
    
    W pewnym momencie, stało się jasne, że komponent \textbf{monasca-log-agent} nie wspiera wpisów w logach, znajdujących się 
    w więcej niż jednej linii. Twórca pracy dyplomowej był jednym z dwóch autorów rozwiązania, które implementowały wyżej 
    opisaną funkcjonalność. Z uwagi na fakt, że \textbf{monasca-log-agent} jest specjalnie skonfigurowaną instancją 
    \textbf{Logstash}, całość wdrożenia została umieszczona w roli\textbf{Ansible} - 
    \href{https://github.com/FujitsuEnablingSoftwareTechnologyGmbH/ansible-monasca-log-agent}{ansible-monasca-log-agent}.
    Główną ideą było dodania dodanie zestawu stałych, opisujących konfiguracją segmentu filtracji w \textbf{Logstash}.

    \subsection{Opis rozwiązania}
    Przykładowy kod filtru, widoczny na listingu \ref{chapter:application_own:own_work:monasca_log_agent:filter_example}, 
    wykorzystuje wtyczkę \href{https://www.elastic.co/guide/en/logstash/current/plugins-filters-multiline.html}{multiline} 
    dla aplikacji \textbf{Logstash}. Najistotniejszym elementem, w zaprezentowanym kodzie, jest sekcja 
    \textit{configuration}. Jest to mapowanie w stosunku 1:1, odpowiadające konfiguracji wspomnianego filtra. Poszczególne 
    elementy oznaczają:
    \begin{itemize}
        \item[pattern] - wyrażenie regularne, wykorzystane do dopasowania linii kończącej lub rozpoczynającej wydarzenia, zapisane w wielu liniach,
        \item[what] - wskazuje, czy wartość \textit{pattern}, powinna być wykorzystana do dopasowania pierwszej lub ostatniej linijki odczytanej
        z monitorowanego pliku,
        \item[negate] - ściśle powiązane z dopasowaniem lub jego brakiem. Jeśli, korzystając z wyrażenia regularnego, nie uda się dopasować linii,
        wartość logiczna takiego faktu, powinna zostać odwrócona. Jeśli wyrażenia zwróci fałsz, ustawienie \textit{negate} na \textit{true}, 
        spowoduje, że faktyczną zwróconą wartością będzie logiczna prawda. Innymi słowy, w omawianym przykładzie, kolejne linie są częścią jednego
        logu dopóki, dopóty wyrażenie regularne zwraca logiczny fałsz. 
    \end{itemize}
    
    \begin{listing}[H]
        \begin{minted}{text}
    
       
           
       
           filter {
               
                   if {{ filter.type }} in [tags] {
                       {{ filter.name }} {
                           
                               {{ key }} => {{ val }}
                           
                        }
                    }
                
            }
    
        \end{minted}
        \caption[Wpisanie filtrów do pliku konfiguracyjnego \textbf{Logstash}]{
            Wpisanie filtrów do pliku konfiguracyjnego \textbf{Logstash}, źródło: \url{https://github.com/FujitsuEnablingSoftwareTechnologyGmbH/ansible-monasca-log-agent/blob/master/templates/agent.conf.j2}}
        \label{chapter:application_own:own_work:monasca_log_agent:filter_applying}
    \end{listing}
    
    Wspomniany filtr z listingu \ref{chapter:application_own:own_work:monasca_log_agent:filter_example}, jest w dalszej części 
    umieszczany w pliku konfiguracyjnym \textbf{Logstash} z wykorzystaniem kodu z listingu 
    \ref{chapter:application_own:own_work:monasca_log_agent:filter_applying}. 
    Wartymi omówienia cechami jest tutaj:
    \begin{itemize}
        \item[elastyczność] - sekcja filtracji zostanie umieszczona w wynikowym pliku, tylko jeśli zostaną zdefiniowane jakiekolwiek filtry,
        \item[spójność] - wspomniana w kodzie z listingu \ref{chapter:application_own:own_work:monasca_log_agent:filter_example} 
        sekcja \textit{configuration}, jest w 100 procentach kompatybilna z dostępnymi dla danego filtra opcjami konfiguracyjnymi,
        \item[możliwość rozszerzenia] - kod jest niewrażliwy na ilość potencjalnych filtrów. Jest on zdolny do umieszczania zarówno konfiguracji dla jednego
        filtra, ale potrafi też wpisać 100 innych. 
    \end{itemize}

    \subsection{Oszacowany wkład}
    W kontekście omawianej funkcjonalności, jaką jest rozwiązanie problemu wielu linii w logach, autor, poniższej
    pracy dyplomowej, miał 50\% wkład w całą implementację. 

\section{Rotacja logów}
    W trakcie tworzenie projektu okazało się, że część z wewnętrznie używanych aplikacji, nie obsługuje automatycznej rotacji 
    logów. Jest to operacja, w wyniku której plik logów, po przekroczeniu pewnej wartości granicznej, ulega archiwizacji. 
    Autor, poniższej pracy dyplomowej, był częścią dyskusji, w wyniku której wybrano aplikację \textbf{logrorate}. Całość 
    implementacji, jest już autorstwa twórcy pracy magisterskiej. Została ona utworzone jako rola \textbf{Ansible}. Dzięki 
    temu uzyskano rozwiązania możliwe do zastosowania na dowolnej liczbie hostów.

\section{Ocena kodu tworzonego w społeczności}
    Autor jest obecny w pracach całej społeczności jako \textbf{review}'er wszystkich zmian do dowolnego projektu, 
    niezależnie od tego, czy proponowany kod jest częścią nowej funkcjonalności, czy też jest to naprawa zauważonych błędów i 
    uchybień. Dzięki temu statusowi, ma on możliwość oceny jakości kodu, wskazywania potencjalnych problemów lub zgłaszania 
    niepoprawnej implementacji w stosunku do komponentów zależnych.
    Proces oceny oparty jest o 3, niezależne, od siebie oceny:
    \begin{itemize}
        \item[code review] - odnosi się jedynie do samego kodu. Ocena pozytywna jest sumą elementów takich jak:
        \begin{itemize}
            \item poprawne wykorzystania języka programowania,
            \item efektywne korzystanie ze struktur danych,
            \item poprawne skorelowania nowego kodu ze starym,
        \end{itemize}
        \item[verified] -  ocena poprawności działania. Innymi słowy, funkcjonalność, którą zmiana, wprowadza lub poprawia, powinna działać
        zgodnie z intencją autora oraz, przynajmniej, nie wpływać negatywnie na inne elementy,
        \item[workflow] - jest oceną dosyć specyficzną z uwagi na istnienie wartości \textbf{-1}, którą autor zmiany może przyznać sobie sam.
        Jest to informacja dla osób weryfikujących, że kod nie jest jeszcze gotowy i wymaga dalszej pracy. Z drugiej strony, ocena pozytywna w tej grupie,
        jest najczęściej ostatnią z przyznawanych, która ostatecznie kwalifikuję zmianą jako zaakceptowaną. W wyniku tego, kod, który był
        jedynie propozycją, wchodzi do głównej gałęzi
    \end{itemize}
    Swój wkład w tym aspekcie, autor stara się okazywać w każdym z projektów spod szyldy \textbf{monasca}.

\section{Nominacja do statusu code-reviewer}
    Status \textbf{code-reviewer} jest, niejako, prestiżową rolą, jaką programista, będący obecny na 
    \url{http://review.openstack.org} może mieć. Co ważniejsze, aby uzyskać ten status należy być nominowanym do niego. Sama 
    nominacja pochodzi od innych członków, zrzeszonych wokół konkretnych projektów i jest ona jedynie warunkiem koniecznym. 
    Finalna decyzja leży po stronie osób nadzorujących działania społeczności.
    Warto dodać, że status \textbf{code-reviewer} upoważnia programistę do finalnej akceptacji lub odrzucenia proponowanych 
    zmian. W innym wypadku, możliwe jest jedynie ocena kodu i ewentualne przetestowanie zmiany. Jednak nie możliwe jest 
    ocenienie jej na oceną, które kwalifikowała by zmianę do akceptacji.