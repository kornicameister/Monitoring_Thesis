\section{Monitorowanie w dużej skali}
\label{chapters:monitoring:at_scale}

Oczywistą cechą dużych systemów informatycznych jest duża liczba komponentów, aplikacji oraz maszyn połączonych ze sobą.
Wszystkie posiadają pewne zadania i każdy z nich może stać się źródłem problemów dla całej infrastruktury. 
Wszelakiego rodzaju incydenty rozumiane są jako straty w dochodach, pośrednie lub bezpośrednie. Jeśli na skutek 
awarii spowolniony zostanie dostęp do strony lub też będzie ona czasowo niedostępna, potencjalny klient
skieruje się do konkurencyjnej witryny. Z tego też powodu, aby uniknąć ujemnego bilansu, organizacje kładą duży nacisk
na wszelkiej maści środki zaradcze działające nieprzerwanie. Podobnie do ilości danych krytycznych dla modelu 
biznesowego i działających w nim usług, również liczba metryk jest tutaj znacząca.

Modelowanie alarmów oraz ich ograniczeń jest wynikiem ciągłego procesu ich doskonalenia, dostosowywania do zmieniającego
się systemu oraz obserwowania metryk jako ciągle ewoluujących modeli danych.
Operatorzy dokładają także wszelkich starań, aby unikać duplikowania się alarmów. Szczególna uwaga
przykładana jest do poprawnej ich organizacji. Sytuacja porównać można do prób unikania nieporządku na półce z książkami.
Zakładając trzy-poziomową organizację książek po gatunku, autorze oraz tytule sytuacja w przypadku alarmów wyglądałaby podobnie.
Pierwszym etapem ich hierarchizacji byłoby przypisanie do pewnej przestrzeni nazw oznaczającej np.
pewną sferę systemu. W dalszej kolejności operator mógłby grupować alarmy w odniesieniu do komponentu (urządzenia lub aplikacji).
Ostatni etap organizacji odnosiłby się już do konkretnego rodzaju monitorowanej własności, na przykład opóźnień w sieci.
Proponowany model hierarchizacji alarmów może wyglądać zupełnie inaczej, jednak wynika z niego, że systematyczne oraz spójne
nazewnictwo daje, w późniejszym etapie, wymierne korzyści. Łatwo, patrząc jedynie na nazwę alarmu zapisanego w powiadomieniu, zorientować się
czego on dotyczy oraz w którym miejscu rozległej infrastruktury system monitorujący zauważył problem. Z drugiej strony jeśli
pewien segment systemu zostałby usunięty, skasowanie skorelowanych z nim alarmów, byłoby znacznie uproszczone.
Organizacja jest pierwszym wyznacznikiem dobrego zarządzania monitoringiem, jednak nie jest jedynym. Metryki spływające
z całego systemu rozumiane są jako model danych, który podlega ciągłym zmianom. Wynika z tego, że ograniczenie, nałożone w początkowym
etapie projektowania, nie jest wartością stałą. Powinno być ono poddane ciągłemu procesowi ewaluacji, aby lepiej 
oddawać charakterystykę systemu. Może to skutkować podnoszeniem pewnych limitów, podczas gdy inne ulegać będą zmniejszeniu. Konieczne 
może okazać się także zwiększanie ilości alarmów w pewnych miejscach, podczas gdy dla innych 
segmentów ich ilość powinna być zmniejszana \cite{monitoring_and_alerting}.