\section{Monitorowanie w dużej skali}
\label{chapters:monitoring:at_scale}

Oczywistą cechą dużych systemów informatycznych jest ilość komponentów, aplikacji oraz maszyn połączonych ze sobą.
Wszystkie z nich posiadają pewne zadania i każdy z nich może stać się źródłem problemów dla całej infrastruktury. 
Wszelakiego rodzaju incydenty rozumiane są jako straty w dochodach, pośrednio lub bezpośrednie. Jeśli na skutek 
awarii, spowolniony zostanie dostęp do strony lub też będzie ona czasowo niedostępny, potencjalny klient
skieruje się do konkurencyjnej witryny. Z tego też powodu, aby uniknąć ujemne bilansu, organizacje kładę duży nacisk
na wszelkiej maści środki zaradcze, działający nieprzerwanie. Podobnie do ilości danych krytycznych dla modelu 
biznesowego i działający w nim usług, również ilość metryk jest tutaj znacząca.

Modelowania alarmów oraz ich ograniczeń jest wynikiem ciągłego procesu ich doskonalenia, dostosowania do zmieniającego
się systemu oraz patrzenie na metryki jako ciągle ewoluujący model danych.
Operatorzy dokładają także wszelkich starań aby unikać duplikowania się alarmów. Szczególna uwaga
przykłada jest do poprawnej ich organizacji. Sytuacja wygląda podobnie do unikania nieporządku na półce z książkami.
Zakładając trzy-poziomową organizację po gatunku, autorze oraz tytule sytuacja w przypadku alarmów wyglądałaby podobnie.
Pierwszym etapem ich hierarchizacji byłoby przypisanie do pewnej przestrzeni nazw oznaczającej na przykład
pewną sferę systemu. W dalszej kolejności, operator, mógłby grupować alarmy w odniesieniu do komponentu (urządzenia lub aplikacji).
Ostatni etap organizacji odnosiłby się już do konkretnego rodzaju monitorowanej własności, na przykład opóźnienia w sieci.
Przykładowy model hierarchizacji alarmów może wyglądać zupełnie inaczej, jednak wynika z niego, że systematyczne oraz spójne
nazewnictwo daje, w późniejszym etapie, wymierne korzyści. Łatwo, patrząc jedynie na nazwę alarmu zapisanego w powiadomieniu, 
czego on dotyczy, w którym miejscu rozległej infrastruktury system monitorujący zauważył problem. Z drugiej strony, jeśli
usunięto by pewien segment systemu, usunięcie skorelowanych z nim alarmów, byłoby znacznie uproszczone.
Organizacja jest pierwszym wyznacznikiem dobrego zarządzania monitoringiem, jednak nie jest jedyna. Metryki spływające
z całego systemu rozumiane są jako model danych, który podlega ciągłym zmianom. Wynika z tego, że ograniczenie, nałożone w początkowym
etapie ich projektowania, nie są wartością stałą. Powinno być one poddane ciągłemu procesowi ewaluacji, aby lepiej 
oddawać charakterystykę systemu. Może to skutkować w podnoszeniu pewnych limitów, podczas gdy inne ulegać będą zmniejszeniu. Konieczne 
może okazywać się także zwiększania ilości alarmów w pewnych miejscach, podczas gdy dla innych 
segmentów ich ilość powinna być zmniejszana \cite{monitoring_and_alerting}.