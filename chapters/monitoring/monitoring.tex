\section{Monitorowanie}
    Monitorowanie, w informatyce, nie różni się szczególnie od monitorowania w innych dziedzinach nauki lub przemysłu. 
    W najszerszym tego słowa znaczeniu, odnosi się ono uzyskania wiedzy na temat tego, co się aktualnie dzieje w systemie.
    Jest to proces, w wyniku którego możliwe jest uzyskania wiedzy na temat obciążanie systemu, rozmiarów oraz przepływu przez niego
    danych. Monitorowanie jest zestawem aplikacji z zakresu przetwarzania danych w czasie rzeczywistym oraz analizy. 
    Głównym zadaniem, jakie przyświeca omawianej gałęzi informatyki, jest alarmowanie administratorów systemu o wszelkich
    odchyleniach od normy, które jeśli się utrzymają, mogą skutkować nawet wyłączeniem monitorowanego
    komponentu przez co niemożliwa będzie normalne praca całego systemu. Najlepszym tutaj przykładem, takiej sytuacji, 
    byłby wzrost ilości niedostarczonych wiadomości w kolejce (przykładowo Kafka lub RabbitMQ). Jeśli, wspomniana kolejka,
    stanowiłaby integralną część systemu, jej zadanie polegałoby na przesyłaniu danych istotnych z punktu
    widzenia użytkownika końcowego, awaria tego typu zatrzymałaby cały system \cite{monitoring_and_alerting}.
    
\subsection{Techniki monitorowania}
    Wyróżnić można dwie podstawowe gałęzie. Podejście \textbf{pasywne} oraz \textbf{aktywne}. Oba z nich wzajemnie się uzupełniają.
    W pierwszym z nich, administrator systemu wyłączony jest z procesu monitorowania stanu dopóki pewien
    punkty krytyczny nie zostanie przekroczony. W tym momencie wystosowywane jest powiadomienie informujące o zaistniałej
    sytuacji. Z drugiej strony, w podejściu aktywnym, administrator może na bieżąco analizować dane spływające z systemu.
    Rozwiązania monitoringu najczęściej dostarczane są razem z programami, pozwalającymi na
    wizualizację zebranych próbek w formie diagramów oraz wykresów. W połączeniu z faktem, że monitorowania odbywa się
    w czasie rzeczywistym, administrator może szybciej zauważyć niepokojące symptomy wskazujące
    na nieprawidłowe działania pewnych komponentów lub systemu jako całości.
    Rozwiązanie pasywne utożsamia się z alarmowaniem, podczas gdy aktywne z monitorowaniem \cite{monitoring_and_alerting}.
    
    \subsection{Monitorowanie - części składowe}
    Niezależnie od przyjętego modelu monitoringu, niezmiennie składa się on zawsze z metryk, danych, gdzie nadrzędną 
    ich własnością jest czas, alarmów, notyfikacji oraz agentów \cite{monitoring_and_alerting}.
    
        \subsubsection{Agenci}
        Proces monitorowania rozpoczyna się od agentów. Agent, jest to specjalny program, działający nieprzerwanie w systemie, którego
        zadaniem jest zbieranie informacji na temat jego działania. Warto w tym miejscu nadmienić, że agent może zbierać zarówno
        dane odnoszące się do pojedynczej aplikacji. Jego zadaniem jest, poza pobraniem danych, przetłumaczenia ich na dane 
        numeryczne. Wynika stąd pewna własność metryk - kolejne pomiary muszą być reprezentowane za pomocą wartości numerycznych.
        Nie jest to problem dla statystyk takich jak aktualnego zużycie procesora lub pamięci. Z drugiej strony, zadaniem agenta
        może być również weryfikacja czy na danej maszynie działa serwer WWW. Poniekąd zachodzi wtedy normalizacja danych. Fakt, że
        serwer działa i jest w stanie obsługiwać żądania, może być reprezentowany za pomocą liczby 1, odwrotnie 0, mogłoby sygnalizować
        przeciwny fakt. Opisany proces może zachodzić w sposób ciągły lub być okresowo wznawiany \cite{monitoring_and_alerting}.
        \newglossaryentry{monitoring_agent}{
            name=Agent,
            description={Aplikacja działająca na monitorowaniej maszynie, odpowiedzialna ze zbieranie metryk}
            } 
        
        \subsubsection{Metryki}
        Dane zebrane od agentów, przechowywane są w formie metryk. Jest to struktura danych optymalizowana pod kątem
        efektywnego przechowywania liczb. Nie miałyby one jednak większego znaczenia bez kontekstu w którym zostały
        zabrane. Dlatego też każda pojedyncza metryka opisana jest zestawem innych wartości dającymi wgląd w takie
        informacje jak na przykład:
        \begin{itemize}
            \item miejsce pochodzenia,
            \item rodzaj metryki,
            \item data zebrania.
        \end{itemize}
        Faktyczna liczba \textit{wymiarów}\footnote{\textbf{Dimensions} - meta dane opisujące zabraną metrykę} może
        zostać dowolnie ustalana. Dzięki nim możliwe jest, w późniejszym czasie, przeglądania zebranych pomiarów
        na dowolnym poziomie agregacji. W dalszym procesie przetwarzania metryki grupowane są w serie czasowe. Ich ilość
        zależy przede wszystkim od rodzaju informacji jaką chce się uzyskać \cite{monitoring_and_alerting}.
        
        \subsubsection{Alarmy}
        Generowane są na podstawie metryk przez oprogramowanie zwane monitorem. Alarm, jeśli zostanie wygenerowany, 
        odnosi się do sytuacji szczególnej, która zaistniała w systemie. Wspomniane aplikacje analizują wygenerowane metryki, 
        porównując otrzymana wartości z dostarczoną im konfiguracją. Bazując na niej, są w stanie stwierdzić, że jeśli
        własność A przekroczyła maksymalną lub minimalną wartość określoną ilość razy, konieczne jest poinformowania o tym
        administratora systemu. Jednocześnie alarm, najczęściej, zapisywany jest do bazy danych. Pozwala to na późniejszą analizę 
        w kontekście historycznym \cite{monitoring_and_alerting}. 
        
        \subsubsection{Powiadomienia}
        Notyfikacje są generowane przez system monitoringu jeśli dany alarm przejdzie w stan aktywny. Mogą przyjąć praktycznie
        dowolną formę, wszystko zależy od pomysłowości programistów stojących za przygotowaniem takiego rozwiązania.
        Najczęściej spotykane jest wysłania wiadomości mailowej na wskazane adres, informując o 
        przekroczeniu pewnych wartości granicznych w określonym komponencie całego systemu. Spotyka się również inne formy:
        \begin{itemize}
            \item[SMS] - na wskazany numer telefonu wysłana jest wiadomość SMS, zawierająca skrócony opis zaistniałego problemu,
            \item[ITS] - z angielskiego - \textbf{Issue Tracking System}. Notyfikacja może przyjąć także bardziej wyrafinowaną formę.
            \textbf{ITS} są specjalnymi rodzajami stron internetowych, w których zgłaszane są wszelkie problemy, jakie napotkano
            podczas użytkowania systemu. Ich nadrzędnym celem jest dostarczenie właściwych rozwiązań administracyjnych, mających
            na celu naprawę zaistniałej sytuacji oraz późniejsze jej przeciwdziałania.
        \end{itemize}
        Mogą się one zarówno wzajemnie wykluczać, jak i uzupełniać. Alarmy mogą również same definiować jaki rodzaj notyfikacji
        powinien zostać w ich przypadku wykorzystany \cite{monitoring_and_alerting}.
        
        \subsubsection{Wizualizacje}
        W formie graficznej, wykresów oraz diagramów, reprezentuje się serie czasowe tworzone na podstawie metryk. 
        Reprezentacja graficzna metryk nie jest bezpośrednio wykorzystywana w przypadku alarmu,
        niemniej nie umniejsza to jej znaczenia. Tego typu funkcjonalność, często dostarczana z oprogramowaniem
        monitorującym, daje wgląd w stan systemu w czasie rzeczywistym. Poszczególne metryki, mogą być
        przeglądane na dowolnym poziomie szczegółowości, te same dane mogą być przedstawione w formie
        wykresu słupkowego lub kołowego, zależnie od potrzeby. Dla operatora jest
        to możliwość dokładnego prześledzenia stanu systemu w zadanym zakresie czasu, przed lub po
        nastąpieniu awarii. Doświadczony administrator, znający charakterystykę systemu, jest w stanie
        wykryć awarię zanim jeszcze zostanie ona wykryta przez system monitorujący. Przykładowo, ciągle
        rosnący zużycie pamięci, może wskazywać, że pewna aplikacja niepoprawnie zarządza przydziałem
        pamięci operacyjnej - nie zwalnia zajętych zasobów. W tym samym momencie, alarm powinien przejść
        w stan aktywny po przekroczeniu co najmniej 80 procent. Administrator jednak, widząc, że zużycia pamięci
        stale wzrastało w przeciągu ostatnich 8 godzin, ma szansę zgłosić ten fakt, a zespół programistów
        może wprowadzić poprawki do oprogramowania usuwające usterkę. Podobne zastosowanie, wizualizację,
        mogą znaleźć dla zużycia przestrzeni dyskowej. Wspomniana sytuacje można przypisać do komponentów
        systemowych - niskiego poziomu w stosunku do aplikacji. Programy jednak również można monitorować,
        i tak więc, niepokojącą oznaką aplikacji internetowej może być ciągle wydłużający się czas 
        ładowania strony, wskazujący na problemy z serwerem lub na niezbyt wydajną architekturę samej aplikacji. 