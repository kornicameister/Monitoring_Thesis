\section{Korzyści monitorowania - zalety i wady}
\label{chapter:monitoring:advantages}

    \subsection{Wczesne wykrycie problemu}
    \label{chapter:monitoring:advantages:early_detection}
    Jednocześnie podstawowa cecha oraz korzyść wynikająca z posiadania systemu monitorowania. 
    Jest to także najbardziej krytyczna funkcja w której stykają się problemy odnoszące się
    do szybkości, dokładności oraz wydajności. Im szybciej administrator zostanie
    powiadomiony o potencjalnym problemie, tym większa szansa, że uda uniknąć się potencjalnej
    awarii systemu lub jej skutki będą zminimalizowane. Niestety, w tym samym czasie, 
    aby system monitorowania mógł dokładnie określić, że obserwowana sytuacja odbiega od normy,
    musi on zebrać więcej niż kilka próbek. Uzyskania zadanej dokładności wymaga czasu, 
    przez co szybkość detekcji spada. Ponadto, podobnie jak każdy program, również procesu
    monitoringu zajmują czas procesora oraz pamięć. Im bardziej skomplikowany jest algorytm
    detekcji tym większych zasobów systemu może on wymagać. Oczywistym jest, że wymaganiem
    stawianym twórców tego typu oprogramowania jest dostarczenie wymaganej funkcjonalności
    bez zbyt dużego wpływu na wydajności docelowego systemu \cite{monitoring_and_alerting}. 
    
    \subsection{Wysoka dostępność}
    \label{chapter:monitoring:advantages:high_availability}
    Zapewnienie ciągłości w świadczeniu usługi jest możliwe dzięki wczesnej detekcji
    wszelkich sytuacji anormalnych (\ref{chapter:monitoring:advantages:early_detection}) w kontekście danego systemu. 
    Termin, mimo ogólności, jest szczególnie istotny w przypadku usług działających w chmurach, dostępnych
    dla szerokiego grona użytkowników. Awaria systemu, w takim przypadku, jest sytuacją niedopuszczalną. Dlatego
    też podejmowane są wszelkie kroki, pozwalające świadczyć usługi bez przerwy. Dzięki możliwości
    monitorowania systemu i wynikającej z tego korzyści jaką niesie wiedza o jego stania, pozwala
    administratorom na aktywne przeciwdziałania tego typu sytuacjom \cite{monitoring_and_alerting}. 
    
    \subsection{Automatyzacja}
    Uzyskania wysokiej dostępności w sposób ręczny, z udziałem czynnika ludzkiego, jest praktycznie niemożliwe.
    Mimo, że to właśnie człowiek jest w stanie zrozumieć przyczyny problemu aby zapobiec ich ponownemu wystąpieniu, 
    nie jest on zdolny do zapewnienia, że system będzie świadczył usługę nieprzerwania. Wysoka dostępność (\ref{chapter:monitoring:advantages:high_availability}) tożsama jest z automatycznym procesem, w wyniku którego
    komponent zgłaszający problem, może zostać wyłączony lub odłączony od reszty systemu i zastąpiony inną instancją.
    Rozwiązania tego typu są najczęściej oparte o technologie takie jak \textbf{Docker}, gdzie gotowy system
    operacyjny wraz z przygotowaną aplikacją zapisane są w kontenerze (podobnym do maszyny wirtualnej), który po 
    uruchomieniu jest natychmiast gotowy do pracy. Nie tylko zastępowania komponentów jest możliwe. Dzięki
    danym napływającym z monitorowania, wskazującym na wyższe obciążania systemu, możliwe jest uruchomienie
    kolejnych instancji, które wspomogą już działające, a zwiększony ruch zostanie rozłożony na więcej elementów,
    przez co sumarycznie poszczególnie instancje zaczną działać pod mniejszym obciążaniem. Odwracając sytuację, jeśli
    dodatkowe instancje nie będą już potrzebne mogą zostać one automatycznie wyłączone. Taki rodzaj pracy, jest również
    znany pod nazwą oprogramowania autonomicznego, zdolnego do samodzielnego skalowania oraz przywracania do stanu równowagi \cite{monitoring_and_alerting}.
    
    \subsection{Dostosowywanie parametrów systemu}
    Informacji pochodzące z alarmu można wykorzystać do zmiany parametrów systemu lub komponentów,
    które powinny działać bardziej wydajnie. Czasami modyfikacja tego typu mogą wynikać z zauważonych
    problemów (\ref{chapter:monitoring:advantages:early_detection}), skutkujących w przepełnianiu
    kolejek lub zajęciu całej pamięci operacyjnej. 