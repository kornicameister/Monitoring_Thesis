\section{As a service - co to oznacza}
\label{chapter:monitoring_architecture:xaas}

\textbf{Anything As A Service} - koncepcja według, której wszystko dostępne, co można dostarczyć
jako usługi informatyczne, można zamówić jako usługę. Pojęcie jest nierozerwalnie związane z chmurami
obliczeniowymi oraz jej możliwościami do dostarczenia dowolnego rodzaju usługi.

Pojęcie zaczęło istnieć niedługo po tym, jak rozpowszechniły się pierwsze aplikacje działające w chmurach, takie jak:
\begin{itemize}
    \item Dropbox,
    \item Gmail,
    \item Office 365,
    \item GDrive + Google Docs
   \end{itemize}
Wspomniane usługi dostarczała funkcji takich jak 
wirtualny dysk twardy, dostępny z każdego urządzenia i miejsca na świecie, rozbudowana skrzynka pocztowa czy też pakiety biurowe bez konieczności instalacji.
Każdą z nich łączy pewna wspólna cecha: \textbf{użytkownik, bez konieczności instalacji dodatkowego oprogramowania lub rozbudowy sprzętu dostaje dostęp do wymienionych funkcji w zerowym czasie}. Innymi słowy, każdy z wymienionych serwisów,
jest wysoce dostępny, czy to przez przeglądarkę internetową, czy to przez aplikację mobilną. Fakt, braku konieczności instalacji dodatkowych programów, mimo że mija się z prawdą w przypadku urządzeń przenośnych, jest pomijalny, z uwagi na niski stopień
trudności dodania nowego programu do smartphone'u.

\subsection{XaaS a CloudComputing}
Oba pojęcia, mimo że dotyczą tego samego, czyli chmur obliczeniowych, znacząco się różnią. \textbf{Cloud computing} - ogólnie, wykorzystywania możliwości
oferowanych przez architekturę chmurz obliczeniowych (wirtualne procesory, pamięć obliczeniowa oraz dyski twarde), jako przenośników usług. Klient, jest w stanie, przykładowo wykupić pewien pakiet, pozwalający mu na dostęp do 5 procesów, 32GB pamięci RAM przez 2 dni. Pomijając cel, jaki mógłby przyświecać potencjalnemu użytkownikowi, należy zwrócić uwagę na koszty. Są one nieporównywalnie niższe od konieczności zakupu fizycznego urządzenia
o tej samej lub porównywalnej mocy. Innymi słowy, \textbf{cloud computing}, z punktu widzenia biznesowego, daje jego dostawcom, produkt - usługę, w postaci zasobów systemowych, które klienci mogą wykorzystywać w dowolnym celu przez określony czas.

Z drugiej strony \textbf{XaaS}, jest to pojęcie, które
opisany wyżej prosty model biznesowy, zawiera w sobie. \textbf{XaaS}, to nie tylko \textit{platformy obliczeniowe} dostępne jako serwis lub też \textit{oprogramowania}.
Należy przez to rozumieć, że wszystko co do tej pory oferowały chmury obliczeniowe, znajduje się pod jednym szyldem i oferowane jest jako pakiet. 
Wraz z infrastrukturą, klient otrzyma również platformę uruchomioną na niej. Na tej platformie, będzie miał on możliwość korzystania z wykupionego oprogramowania.

\subsection{Implikacje XaaS}
Idea, jaka przyświeca omawianemu pojęcia, wykracza poza ramy informatyki. W Kalifornii, funkcjonuje
przedsiębiorstwo oferujące energię słoneczną jako usługę. Głównym założeniem, jest nie sprzedawania
urządzeń, które przetwarzają energię, pochodzącą ze słońca, na ciepło lub prąd. Osoby, stojące za tym pomysłem, sprzedają ilość energii oraz ciepła, jakie 
panele mogą wyprodukować w określonym czasie. Klient nie jest obciążany kosztami zakupu oraz instalacji
potrzebnych urządzeń. Płaci on za końcowy produkt. Wyraźnie widać, w tym miejscu, nawiązanie do 
podobnych usług w chmurach obliczeniowych. Przykładowo, Amazon, oferuje wirtualne maszyny o określonej mocy na określony czas. W tym wypadku, 
użytkownik płaci jedynie za okres, na który wykupił daną instancję lub instancje \cite{linkedin_anything_as_a_service}. Mimo, że opisywany przykład,
nie nosi znamion jakiegokolwiek związku z chmurą obliczeniową, jest to jedynie złudzenia. Aby usługodawca mógł kontrolować, jak dużo energii panele
już wygenerowały, czyli jak duży procent wykupionej usługi został już zrealizowany, potrzebuje on do tego odpowiednich czujników. Zbieranie oraz
przetwarzanie zebranych danych wymaga oczywiście odpowiedniego oprogramowania działającego na komputerze. Ponownie więc firma ma możliwość
wykupienia odpowiednich zasobów, aby zrealizować ostatecznie swoją usługę. 