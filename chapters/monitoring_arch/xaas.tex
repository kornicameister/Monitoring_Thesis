\section{As a service - co to oznacza}
\label{chapter:monitoring_architecture:xaas}

\textbf{Anything As A Service} - koncepcja, według której wszystko dostępne, co można dostarczyć
jako usługi informatyczne, można zamówić jako usługę.
%a co autor miał na myśli powyżej?
 Pojęcie jest nierozerwalnie związane z chmurami
obliczeniowymi oraz ich możliwościami dostarczenia dowolnego rodzaju usługi.

Termin ten zaczęło istnieć niedługo po tym jak rozpowszechniły się pierwsze aplikacje działające w chmurach, takie jak:
\begin{itemize}
    \item Dropbox,
    \item Gmail,
    \item Office 365,
    \item GDrive + Google Docs.
   \end{itemize}
Wspomniane usługi dostarczają funkcji tj. wirtualny dysk twardy, dostępny z każdego urządzenia i miejsca na świecie,
rozbudowana skrzynka pocztowa czy też pakiety narzędzi biurowych dostępne bez konieczności instalacji.
Każdą z nich łączy pewna wspólna cecha: \textbf{użytkownik bez konieczności instalacji dodatkowego oprogramowania lub rozbudowy sprzętu dostaje dostęp do wymienionych funkcji w zerowym czasie}. Innymi słowy każdy z wymienionych serwisów
jest stale dostępny, czy to przez przeglądarkę internetową, czy to przez aplikację mobilną. Brak konieczności instalacji dodatkowych programów, mimo że nie do końca jest prawdą w przypadku urządzeń przenośnych, jest pomijalny z uwagi na niski stopień
trudności dodania nowego programu do urządzenia przenośnego. 

\subsection{XaaS a CloudComputing}
Oba pojęcia, mimo że dotyczą tego samego - chmur obliczeniowych - znacząco się różnią. \textbf{Cloud computing} - ogólnie wykorzystywania funkcjonalność,
oferowana przez architekturę chmur obliczeniowych (wirtualne procesory, pamięć obliczeniowa oraz dyski twarde), 
jako przenośnik usług. Klient jest w stanie, przykładowo, wykupić pewien pakiet pozwalający mu na dostęp do 5 procesów, 
32GB pamięci RAM przez 2 dni. Pomijając cel, jaki mógłby przyświecać potencjalnemu użytkownikowi, należy zwrócić uwagę na koszty. 
Są one nieporównywalnie niższe w stosunku do konieczności zakupu fizycznego urządzenia
o tej samej lub podobnej mocy. Innymi słowy \textbf{cloud computing},
z punktu widzenia biznesowego, daje jego dostawcom produkt - usługę, w postaci zasobów 
systemowych, które klienci mogą wykorzystywać w dowolnym celu przez określony czas.

Z drugiej strony \textbf{XaaS} jest to pojęcie, które
opisany wyżej prosty model biznesowy zawiera w sobie. \textbf{XaaS} to nie tylko \textit{platformy obliczeniowe} dostępne jako serwis lub też \textit{oprogramowanie}.
Należy przez to rozumieć, że wszystko co do tej pory oferowały chmury obliczeniowe, znajduje się pod jednym szyldem i oferowane jest jako pakiet. 
Wraz z infrastrukturą klient otrzyma również platformę uruchomioną na niej. Na tej platformie będzie miał on możliwość korzystania z wykupionego oprogramowania.

\subsection{Implikacje XaaS}
Idea jaka przyświeca omawianemu pojęciu wykracza poza ramy informatyki. W Kalifornii funkcjonuje
przedsiębiorstwo oferujące energię słoneczną jako usługę. Głównym założeniem nie jest jednak sprzedawanie
urządzeń, które przetwarzają energię pochodzącą ze słońca na ciepło lub prąd. Osoby stojące za tym pomysłem sprzedają ilość energii oraz ciepła, jakie 
panele mogą wyprodukować w określonym czasie. Klient nie jest obciążany kosztami zakupu oraz instalacji
potrzebnych urządzeń. Płaci on za końcowy produkt. Wyraźnie widać w tym miejscu nawiązanie do 
podobnych usług w chmurach obliczeniowych. Przykładowo Amazon oferuje wirtualne maszyny o określonej mocy na określony czas. W tym wypadku
użytkownik płaci jedynie za okres, na który wykupił daną instancję lub instancje. Mimo, że opisywany przykład,
nie nosi znamion jakiegokolwiek związku z chmurą obliczeniową, jest to jedynie złudzenie. Aby usługodawca mógł kontrolować jak dużo energii panele
już wygenerowały, czyli jak duży procent wykupionej usługi został już zrealizowany, potrzebuje on do tego odpowiednich czujników. Zbieranie oraz
przetwarzanie zebranych danych wymaga oczywiście odpowiedniego oprogramowania działającego na komputerze. Ponownie więc firma ma możliwość
wykupienia odpowiednich zasobów, aby zrealizować ostatecznie swoją usługę \cite{linkedin_anything_as_a_service}. 