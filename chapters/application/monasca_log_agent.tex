\section{monasca-log-agent - zbieranie logów}
\label{chapter:monasca:monasca_log_agent}

    \subsection{Znaczenie komponentu}
    \textbf{monasca-log-agent} jest praktyczną implementacją koncepcji monitorowania aplikacji opartej o agentów.
    Agent zlokalizowany jest zatem jak najbliżej danych, które ma zbierać, a następnie przetworzyć i wyniki
    przesłać dalej. Jest to efektywne podejście z uwagi na wolumin danych. Tempo oraz ilość generowanych logów
    może być znaczące, dlatego też pobieranie danych poprzez sieć istotnie wydłużałoby całkowity czas 
    wszystkich operacji, jakie agent musi wykonać.
    
    Agent został oparty o jeden z elementów stosu technologicznego \textbf{ELKStack} - \textbf{Logstash}.
    Do zadań agenta należy:
    \begin{itemize}
        \item obserwacja wskazanego pliku z logami,
        \item wykrywanie i łączenie wpisów składających się z wielu linii,
        \item dodawanie informacji o tenancie chmury obliczeniowej,
        \item wskazywanie na typ aplikacji, z której logi są zbierane,
        \item wysyłanie danych do \textbf{monasca-log-api}. (\ref{chapter:monasca:monasca_log_api})
    \end{itemize}
    
    Konfiguracja agenta odbywa się w sposób deklaratywny, poprzez plik konfiguracyjny. Poprawne
    skonfigurowanie sprowadza się do:
    \begin{itemize}
        \item wskazania źródła danych,
        \item określenia miejsca docelowego, do którego dane mają zostać wysłane.
    \end{itemize}
    W przypadku monitorowania logów, konfiguracja jest bardziej rozbudowana.
    Agent musi zwrócić szczególną uwagę na problem wpisów, które znajdują się w więcej niż jednej linii.
    Najczęściej ten problem objawia się, gdy aplikacja zgłasza błędy, a w logach pojawią się rekordy, które
    te problemy opisują. Do opisu błędu (w kontekście logowania) dodawany jest tak zwany \textbf{stacktrace}.
    Zawiera on szczegółowy zrzut wywołań kolejnych funkcji w różnych modułach, w wyniku których aplikacja
    nie wykonała poprawnie danego zadania. \textbf{Stacktrace}, dla łatwości odczytu przez człowieka,
    jest najczęściej reprezentowany w postaci wpisu zajmującego wiele linii. Z punktu widzenia domyślnej
    konfiguracji \textbf{logstash} kolejne linie to zupełnie odrębne wydarzenia. Dla administratora systemu jest
    to jednak logicznie spójna całość, którą on sam analizowałby w takim kontekście. Agent musi więc wiedzieć
    w jaki sposób połączyć oddzielnie linie w jedno i w takiej formie przesłać je do wyjścia.
    
    \subsection{Konfiguracja agenta}
        \label{chapter:monasca:monasca_log_agent:configuration}
        \begin{listing}
            \inputminted[
            fontfamily=monospace,
            fontsize=\scriptsize,
            obeytabs=true,
            samepage=true
            ]{rb}{listingings/monasca_log_agent.conf}
            \label{chapter:monasca:monasca_log_agent:configuration:example_code}
            \caption[Przykładowa konfiguracja monasca-log-agent]{Przykładowa konfiguracja monasca-log-agent, źródło: opracowanie własne}
        \end{listing}
        
        \subsubsection{Obserwacja wskazanego pliku z logami}
            Do konfiguracji logstash'a przekazywana jest ścieżka lub wyrażenie regularne wskazujące
            na plik lub listę plików, z których odczytywane mają być rekordy. \textbf{Logstash}
            wykonuje to zadanie podobnie do Linuksowego narzędzia \textbf{tail}. Kolejne zdarzenia w danej 
            instancji agenta to kolejne linie w obserwowanych plikach, pojawiające się na ich końcu. 
        
        \subsubsection{Wykrywanie i łączenie wpisów składających się z wielu linii}
            Problem wykrywania i łączenia wpisów składających się z wielu linii jest rozwiązany z użyciem następującego algorytmu:
            \begin{enumerate}
                \item określenie jaki typ aplikacji jest monitorowany,
                \item utworzenie adekwatnego wyrażenia regularnego opisującego początek lub koniec wydarzenia,
                \item określenie kierunku łączenia kolejnych rekordów.
            \end{enumerate}
            
            \begin{listing}
                \inputminted[
                fontfamily=monospace,
                fontsize=\scriptsize,
                obeytabs=true,
                samepage=true
                ]{yaml}{listingings/monasca_log_agent_multiline.yml}
                \label{chapter:monasca:monasca_log_agent:configuration:multiline_code}
                \caption[Wsparcie dla wielu linii]{Wsparcie dla wielu linii, źródło: 
                    \url{https://github.com/FujitsuEnablingSoftwareTechnologyGmbH/ansible-monasca-log-agent/blob/master/vars/main.yml}}
            \end{listing}
            
        \subsubsection{Dodawanie informacji o tenancie chmury obliczeniowej}
            Agent, jak zostało to wcześniej wspomniane, działa na konkretnej maszynie. Jedną z jej własności jest
            informacja o użytkowniku - tenancie - unikatowy numer UUID. W przypadku logowania jako serwisu,
            szczególnie istotna jest wiedza właśnie o użytkowniku chmury, na rzecz którego dane logi zostały
            zebrane. \textbf{monasca-log-agent} rozwiązuje ten problem z wykorzystaniem biblioteki 
            \textbf{Keystone}\footnote{Keystone - aplikacja mająca za zadanie rozwiązać problem autentykacji oraz autoryzacji 
                w chmurach obliczeniowych Openstack}. 
            Agent tuż przed przesłaniem logu do \textbf{monasca-log-api} wykonuje procedurę logowania do chmury
            używając, dostarczonych w konfiguracji, nazwy użytkownika oraz hasła. Jako wynik, jeśli logowanie 
            zakończyło się sukcesem, otrzymuje token oraz numer tenanta. Te informacje dołączane są do logu.
            
        \subsubsection{Wskazania na aplikację}
            Informacja o logu zostaje wzbogacona o dane, które pozwalają określić typ oraz samą aplikacją.
            Jest to informacja przydatna, ponieważ jeden program może zapisać logi do więcej niż jednego pliku.
            Przykładowo taka sytuacja występuje w przypadku serwera Tomcat. Oprócz rekordów pochodzących
            z uruchomionej aplikacji możliwe jest także zbieranie danych na temat wszystkich możliwych
            żądań, jakie zostały odebrane przez serwer, a także zebranie logów związanych z autoryzacją użytkowników, jeśli
            takowa odbywa się ze wsparciem Tomcat'a.
            
         \subsubsection{Wysyłanie danych do monasca-log-api}
             Wysyłanie danych do monasca-log-api jest poprzedzone logowaniem do chmury obliczeniowej, jeśli nie
             zostało ono wcześniej wykonane lub jeśli token wygasł. Po udanej autentykacji, kolejne rekordy
             z logu przesyłane są pojedynczo do serwera, gdzie działa \textbf{monasca-log-api}.