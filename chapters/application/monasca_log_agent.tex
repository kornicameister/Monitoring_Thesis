\section{monasca-log-agent - zbieranie logów}
\label{chapter:monasca:monasca_log_agent}

    \subsection{Znaczenie komponentu}
    \textbf{monasca-log-agent} jest praktyczną implementacją koncepcji monitorowania aplikacji opartej o agentów.
    Agent zlokalizowany jest więc najbliżej danych, które ma zbierać, a które ma za zadanie przetworzyć i wyniki
    przesłać dalej. Jest to efektywne podejście z uwagi na wolumin danych. Tempo oraz ilość generowanych logów
    może być znacząca, dlatego też pobieranie danych poprzez sieć, znacząco wydłużałoby całkowity czas 
    wszystkich operacji, jakie agent musi wykonać.
    
    \todo[inline]{schemat rozmieszczania agentów, linie schodzą się do monasca-log-api}
    
    Agent został oparty o jeden z elementów stosu technologicznego \textbf{ELKStack} - \textbf{Logstash}.
    Do zadań agenta należy:
    \begin{itemize}
        \item obserwacja wskazanego pliku z logami,
        \item wykrywania i łączenie wpisów składających się z wielu linii,
        \item dodawania informacji o tenancie chmury obliczeniowej,
        \item wskazywania na typ aplikacji, z której logi są zbierane,
        \item wysyłania danych do \textbf{monasca-log-api} \ref{chapter:monasca:monasca_log_api}
    \end{itemize}
    
    Konfiguracja agenta odbywa się w sposób deklaratywny, poprzez plik konfiguracyjny. Poprawne
    skonfigurowanie sprowadza się do:
    \begin{itemize}
        \item wskazania źródła danych,
        \item określenia miejsca docelowego, do którego dane mają zostać wysłane
    \end{itemize}
    W przypadku monitorowania logów, konfiguracja jest bardziej rozbudowane.
    Agent musi zwrócić szczególną uwagę na problem wpisów, które znajdują się w więcej niż jednej linii.
    Najczęściej ten problem objawia się, gdy aplikacja zgłasza błędy, a logach pojawią się rekordy, które
    te problemy opisują. Do opisu błędu (w kontekście logowania) dodawany jest tak zwany \textbf{stacktrace}.
    Zawiera on szczegółowy zrzut wywołań kolejnych funkcji, w różnych modułach, w wyniku których aplikacja
    nie wykonała poprawnie danego zadania. \textbf{Stacktrace}, dla łatwości odczytu przez człowieka,
    jest najczęściej reprezentowany w postaci wpisu zajmującego wiele linii. Z punktu widzenia domyślnej
    konfiguracji \textbf{logstash} kolejne linii to zupełnie odrębne wydarzenia. Dla administratora systemu jest
    to jednak logicznie spójna całość, którą on sam analizował by w takim kontekście. Agent musi więc wiedzieć,
    w jaki sposób połączyć oddzielnie linie w jedno i w takiej formie przesłać je do wyjścia.
    
    \subsection{Problem wielu linii}
    
    \subsection{Konfiguracja agenta}
        \todo[inline]{przykładowa konfiguracja}
        
        \subsubsection{Obserwacja wskazanego pliku z logami}
            Do konfiguracji logstash'a przekazywana jest ścieżka lub wyrażenia regularnego wskazującego
            na, odpowiednio, plik lub listę plików, z których odczytywana mają być rekordy. \textbf{Logstash}
            wykonuje to zadanie podobnie do Linuksowego narzędzia \textbf{tail}. Kolejne zdarzenia w danej 
            instancji agenta, to kolejne linii w obserwowanych plikach, pojawiające się na ich końcu. 
        
        \subsubsection{Wykrywania i łączenie wpisów składających się z wielu linii}
            Problem ten rozwiązany jest z użyciem następującego algorytmu:
            \begin{enumerate}
                \item określenie jaki typ aplikacji jest monitorowany,
                \item utworzenia adekwatnego wyrażenia regularnego opisującego początek lub koniec wydarzenia,
                \item określenie kierunku łączenia kolejnych rekordów
            \end{enumerate}
            \todo[inline]{Ansible i przygotowane szablony dla multiline}
            
        \subsubsection{Dodawania informacji o tenancie chmury obliczeniowej}
            Agent, jak zostało to wcześniej wspomniane, działa na konkretnej maszynie. Jedną z jej własności jest
            informacja o użytkowniku - tenancie - unikatowy numer UUID. W przypadku logowania jako serwisu,
            szczególnie istotna jest wiedza właśnie o użytkowniku chmury, na rzecz którego, dane logi zostały
            zebrane. \textbf{monasca-log-agent} rozwiązuje ten problem z wykorzystaniem biblioteki \textbf{keystone}
            \footnote{Keystone - aplikacja mająca za zadania rozwiązać problem autentykacji oraz autoryzacji w
                chmurach obliczeniowych Openstack}. 
            Agent, tuż przed przesłanie logu do \textbf{monasca-log-api} wykonuje procedurę logowania do chmury
            używając, dostarczonych w konfiguracji, nazwy użytkownika oraz hasła. Jako wynik, jeśli logowanie 
            zakończyło się sukcesem, otrzymuje token oraz numer tenanta. Dołączony jest on do logu.
            
        \subsubsection{Wskazania na aplikację}
            Informacja o logu zostaje wzbogacona o dane, które pozwalają określić typ oraz samą aplikacją.
            
         \subsubsection{Wysyłanie danych do monasca-log-api}
             Wysyłania danych do monasca-log-api jest poprzedzone logowanie do chmury obliczeniowej, jeśli nie
             zostało ono wcześniej wykonane lub jeśli token wygasł. Po udanej autentykacji, kolejne rekordy
             z logu przesyłane są pojedynczo do serwera, gdzie działa \textbf{monasca-log-api}.