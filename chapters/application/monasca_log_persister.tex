\section{monasca-log-persister - zapisanie logów do bazy danych}
\label{chapter:monasca:monasca_log_persister}

\textbf{monasca-log-persister} jest ostatnim etapem przed umieszczeniem logów w bazie danych \textbf{ElasticSearch}.
Implementacja została oparta o program \textbf{Logstash}. Prócz zapisu danych do miejsca docelowego, zadaniem
aplikacji jest uzupełnienie logów do postaci, która będzie zrozumiała dla \textbf{ElasticSearch}. 

\begin{listing}
    \inputminted[
    fontfamily=monospace,
    fontsize=\scriptsize,
    obeytabs=true,
    samepage=true
    ]{json}{listingings/persister_input_event.json}
    \label{chapter:monasca:monasca_log_persister:input_message}
    \caption[Przykładowe dane wejściowe dla monasca-log-persister]{
        Przykładowe dane wejściowe dla monasca-log-persister, źródło: opracowanie własne}
\end{listing}

Temu procesowi poddawane są pola takie jak:
\begin{itemize}
    \item[timestamp] - domyślne pole \textbf{@timestamp} generowane jest dla zdarzenia przez \textbf{Logstash}. W przypadku
    logu ważne jest aby ten czas oraz czas, w którym rekord został wygenerowany były tożsame. Dlatego też, program, jeśli
    odnajdzie takową informację w strukturze opisującą log, używa jej jako czasu utworzenia struktury wydarzenia przez instancję
    \textbf{monasca-log-persister}. Dzięki temu, to moment wygenerowania rekordu, jest realnym czasem, według której, w dalszej
    kolejności kolejne pozycje będą sortowane w Kibana.
    \item[creation-time] - znacznik czasowy, wskazujący na odebrania i przetworzenia rekordu przez \textbf{monasca-log-api}.
    \item[dimensions] - \textbf{monasca-log-persister} przenosi wszystkie wartości ustawione w tym polu, do poziomu struktury
    jaką jest zdarzenie.
    \item[application-type] - pole wskazujące na aplikację, która wygenerowała log, podobnie jak \textit{dimensions}, jest przenoszone
    do samego wydarzenia
\end{itemize}

Podobnej operacji podlegają także następujące pola, które oryginalnie zawarte są w logu:
\begin{itemize}
    \item[message] - oryginalna wiadomość, która została pobrana przez \textbf{monasca-log-agent},
    \item[level] - poziom logu przenoszony jest do poziomu wydarzenia jako \textbf{log\_level},
    \item[tenant\_id] - identyfikator użytkownika,
    \item[region] - informacja o regionie, w którym znajdowała się maszyna, a z której pochodzi log,
    \item[path] - ścieżka do pliku, z której pochodzą dane
\end{itemize}