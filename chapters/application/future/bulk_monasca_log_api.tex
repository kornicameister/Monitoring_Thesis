\subsection{Wiele logów dla monasca-log-api (bulk-mode)}
\label{chapter:application_own:plans:bulk_monasca_log_api}

W pierwszych słowach warto nadmienić, że opisana w poniższym rozdziale funkcjonalność, oryginalnie została 
zaproponowana zanim zespół, którego autor pracy dyplomowej jest członkiem, rozpoczął właściwie prace
nad implementacją kolejnych elementów, wchodzących w skład całego systemu. Na obecną chwilę, z ideą 
dodania trybu \textbf{bulk} do \textbf{monasca-log-api}, wyszła firma \textbf{TSV}, które zauważyła brak, tej
ważnej funkcjonalności. Interesującym jest, że ten sam fakt, autor oraz inni członkowie zespołu, zauważyli
w tym samym momencie. 

Ideę całego rozwiązania będzie możliwość przesyłania z \textbf{monasca-log-agent} wielu zebranych logów
w jednym żądaniu. Poprzez to, cały proces przetwarzania, zostanie przyspieszony i ostatecznie, logi
będzie można analizować szybciej. Szybsza analiza wiąże się również z szybszym zapisanie logów w centralnej lokalizacji,
a tym samym zabezpieczenie całego systemu przed ich utratą. Autor pracy dyplomowej jest jedną z dwóch osób, które będą
współpracowały z firmą TSV na wymienionych poniżej płaszczyznach:
\begin{itemize}
    \item akceptacja zaproponowanego przez TSV planu implementacji,
    \item zaproponowanie koniecznych zmian, które należy wprowadzić do istniejącej implementacji \textbf{monasca-log-api},
    w kontekście adresów, pod którymi dostępna będzie logika, a na które będzie można wysyłać logi,
    \item ocenie i testowaniu zmian w kodzie, dążących do rozszerzenia \textbf{monasca-log-api},
    \item finalnej akceptacji zmian
\end{itemize}