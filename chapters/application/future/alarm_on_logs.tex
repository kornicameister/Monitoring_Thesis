\section{Alarmy a logi}
\label{chapter:application_own:plans:alarm_on_logs}

Jednym z założeń projektu \textbf{monasca} jest powiadamiania użytkownika o 
sytuacjach krytycznych zaistniałych w systemie. Funkcjonalność realizowana jest na podstawie
metryk, zbieranych z całego systemu. Przekroczenie, założonych przez administratora, wartości
krytycznych, skutkuje wygenerowaniem alarmu. Jednak idea ta, nie może zostać
tak łatwo przełożona na logi. Problemem jest przedstawienie logów w postaci metryk - czyli wartości
numerycznych. W przeciwieństwie do danych dotyczących, chociażby, zużycia przestrzeni dyskowej lub, podlegającemu większej
fluktuacji, obciążeniu procesora, logi stanowią element zbioru metryk nieciągłych \footnote{Sparse metric - metryki okresowe,
zwracające wartości w nieregularnych odstępach czasu}. 

Częścią rozwiązania problemu, byłaby modyfikacja aplikacji \textbf{monasca-thresh}, odpowiedzialnej za faktyczne generowania 
alarmów. Na obecną chwilę, jeśli dla danej metryki, brak jest danych, monitor zdefiniowany dla danego alarmu powoduje jego 
wejście w stan \textbf{UNDETERMINED}. W przypadku logów, nie jest to jednak sytuacja oczekiwana. Przykładowo, brak wystąpień 
błędów dla danej aplikacji, powinien faktycznie oznaczać, że aplikacja lub system pracuje poprawnie i potencjalny operator nie 
ma powodów do niepokoju. Innymi słowy, odpowiadałoby to stanowi \textbf{OK}. Dopiero, co najmniej jedna wartość wykraczająca 
poza ramy danego alarmu, powinna spowodować jego przejście w stan wysoki oraz wygenerowanie powiadomienia lub powiadomień.

Funkcjonalność opisana powyżej jest obecnie w fazie projektowania i podlega dyskusji między członkami całej społeczności.
Została ona omówiona podczas \textbf{Monasca Mid-Cycle}, spotkania wszystkich programistów zrzeszonych wokół projektu
\textbf{monasca}, które odbywało się 3 oraz 4 lutego 2016 roku. Autor poniższej pracy dyplomowej, był aktywnie włączony
w dyskusję, gdzie następujące aspekty zostały omówione:
\begin{itemize}
    \item wprowadzenia nowej zmiennej \textbf{period} do metryk opisującej, jak długi okres czasu musi upłynąć aby
    zmienił się stan alarmu powiązanego z metryką. Dla omawianej koncepcji (alarmy a logi) przyjęte zostało użycie wartości
    \textbf{-1} aby wskazać, że dane metryka jest nieokresowa,
    \item sposób dostarczenie nowych metryk do systemu \textbf{monasca}. Miałaby one trafiać, poprzez kolejkę Kafka,
    do aplikacji \textbf{monasca-thresh},
    \item modyfikacji \textbf{monasca-thresh} aby rozumiała metryki okresowe oraz nieokresowe
\end{itemize}