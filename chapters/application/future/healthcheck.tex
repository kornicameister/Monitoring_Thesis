\newglossaryentry{healthcheck}{
    name={Healthcheck},
    description={
        Specjalny adres, dostępny w \textbf{REST}'owym API, który informuje odpytującego o stanie aplikacji. Zależnie
        od implementacji, można wyróżnić co najmniej dwa typy tego rodzaju komponentów:
        \begin{itemize}
            \item[proste] - informujące jedynie o tym, czy API jest dostępne i można wykonywać żądania,
            \item[złożone] - oferujące tą samą funkcjonalność co \textit{proste} oraz, dodatkowo,
            weryfikujące stan zależnych serwisów, jak chociażby baza danych.
        \end{itemize}
    }
}
\newglossaryentry{loadbalancer}{
    name={Load Balancer},
    description={
        Rodzaj aplikacji, najczęściej znajdujący się przed klastrem serwerów. Jego nadrzędnym zadaniem jest
        dystrybuowanie ruchu do poszczególnych węzłów według zadanego algorytmu. Operacja ta ma na celu,
        równomiernie rozłożenie obciążenia, które generowane jest przez klientów, łączących się do klastra
    }
}

\subsection{Healthcheck}
\label{chapter:application:plans:healthcheck}

Mechanizmy oferowane przez \glslink{healthcheck}{\textbf{healtcheck}'i} dostępne są, na obecną chwilę, jedynie
w następujących aplikacjach: \textbf{monasca-api} oraz \textbf{monasca-log-api}. Dodatkowo, są to wersje napisane w języku
Java. Biorąc pod uwagę trend w społeczności \textbf{monasca} dążący do przejścia na wersje napisane w języku Python,
autor, poniższej pracy dyplomowej, stworzył propozycję implementacji na potrzeby \textbf{monasca-log-api}. Dzięki
temu, możliwe jest wysyłanie żądań do serwera, na które odpowiedzieć on może w następujący sposób:
\begin{itemize}
    \item[HTTP 204] - oznacza, że \textbf{monasca-log-api} działa oraz \textbf{Kafka}, które wykorzystywane jest
    do przysyłania danych do \textbf{monasca-log-transformer} jest również obecne w sieci, a wymagane \glslink{kafka_topic}{topic'i} są
    utworzone
    \item[HTTP 503] - komponent zależny - kafka - jest niedostępna lub niepoprawnie skonfigurowana,
    \item[HTTP 404] - \textbf{monasca-log-api} jest niedostępny
\end{itemize}

Całość została oparta o bibliotekę \textbf{oslo.middleware}. Warto dodać, że ta oraz inne biblioteki, których nazwa
rozpoczyna się od \textbf{oslo.} są, praktycznie, nieodłączną częścią chmury obliczeniowej \textbf{OpenStack}. Z tego też powodu,
są wykorzystywane w bardzo dużej liczbie projektów i to nie tylko tych, które stanowią główną część \textbf{OpenStack} (jak chociażby
\textbf{Horizon czy \textbf{Nova}}).

Dodatkowo autor pracy dyplomowej, zajął się zweryfikowaniem możliwych alternatyw dla omawianego
rozwiązania. Problematyczna okazuje się, bowiem, jedna różnica między wersjami napisanymi w językach \textbf{Python} oraz \textbf{Java}.
Implementacja oparta o język \textbf{Java} wykorzystuję bibliotekę \textbf{DropWizard}, która domyślnie wspiera funkcjonalność określaną
nazwą \textit{healthcheck}. Dodatkowo, serwer uruchamia kolejną (wbudowaną) usługę dostępną pod innym portem.
Jest to duża zaleta w momencie kiedy \textbf{API} umieszcza się wraz z innymi jego instancjami, a gdzie wszystkie umieszczone są
za specjalnym koordynatorem ruchu - \glslink{loadbalancer}{\textbf{loadbalencer}'em}. Posiadanie \textbf{healtcheck} poza głównym API, pozwala
na wydzielenie tych żądań spoza zakresy pracy \textbf{loadbalencer}. Niestety, biblioteki dostępne w \textbf{Python} nie dają takich możliwość.
Również próby uruchomienia wbudowanego serwera WSGI, działającego na innym porcie, okazały się nieskuteczne. Dużym problem okazał się
sposób pracy serwera \glslink{wsgi}{WSGI} wybranego przez społeczność - Gunicorn. W momencie załadowania kompletnej aplikacji, Gunicorn
wchodzi w pętlę, której nadrzędny wątek jest blokujący i wznawiany w momencie otrzymania żądania. Przez to uruchomiony wewnętrznie serwer, nie
pozwala na uruchomienie właściwej aplikacji. Rozwiązaniem wydawałoby się więc wystartowanie go wywnętrz oddzielnego procesu, działającego równoległe
do głównego wątku API. Niestety, przy tym podejściu, niemożliwe jest poprawne wyłączenie zarówno wątku \textbf{healthcheck} oraz \textbf{API}.
Biorąc pod uwagę te problemy oraz możliwości oferowane przez \textbf{oslo.middleware}, omawiana funkcjonalność została zrealizowana jako filtr,
uruchamiany przed faktyczną logiką biznesową. 

Propozycja implementacji dostępna jest pod adresem \url{https://review.openstack.org/#/c/249685/} i oczekuje na akceptację 
społeczności lub dalsze usprawnienia.