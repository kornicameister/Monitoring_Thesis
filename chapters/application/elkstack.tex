\section{ELKStack}
\label{chapter:application:elkstack}

\textbf{ELKStack} - akronim opisujący stos technologi składający się z 3 elementów:
\begin{itemize}
    \item[ElasticSearch] - pełnotekstowy silnik wyszukiwania,
    \item[Logstash] - scentralizowane przetwarzania danych,
    \item[Kibana] - interfejs graficzny silnika \textbf{ElasticSearch}
\end{itemize}

\subsection{Logstash}
    Logstash jest wysoce elastycznym narzędziem, nie tylko w kontekście architektury \textbf{ELKStack}.
    Możliwe jest skonfigurowanie go, aby działał jako kolektor danych, jako ich filtr czy też
    transformator. Składa się on z 3 elementów:
    \begin{itemize}
        \item sekcji wejścia,
        \item sekcji przetwarzania,
        \item sekcji wyjścia
    \end{itemize}
    Każda z nich może składać się z więcej niż jednego bloku, opisującego jej zadania.
    Innymi słowy ilość wejść jest nieograniczona tak samo jak możliwości przetworzenia
    odebranych danych oraz ostatecznie wysłania ich do więcej niż jednej lokalizacji.
    Ponadto dla każdej z nich możliwe jest napisania własnego modułu lub wykorzystanie
    jednego z, ciągle rosnącej listy, publicznie dostępnych.
    
    Jest to ponadto rozwiązanie szybkie oraz łatwo skalowalne. Uruchomienie kolejnej
    instancji sprowadza się do przygotowania odpowiedniego pliku konfiguracyjnego
    i dostarczenia go jako argumentu wejściowego do pliku wykonywalnego aplikacji logstash.
    
    \todo[inline]{Opisać korzyści ???}

\subsection{ElasticSearch}
    Pełnotekstowych, indeksowany silnik wyszukiwania. Jednak \textbf{ElasticSearch} jest także
    nierelacyjną bazą danych opartą o koncepcję dokumentów. Możliwość przechowywania oraz wyszukiwania
    informacji pośród zgromadzonych danych, które mogą mieć zarówno ustaloną strukturę, jak i  luźną, jest
    szczególnie istotna dla zarządzania logami. Tym co wyróżnia każdy z nich jest informacja, wiadomość
    zawarta w kolejnych rekordach, a która jest inna dla każdego z nich. 
    
    ElasticSearch oferuje możliwość pełnotekstowej wyszukiwarki na przechowywanych danych, dzięki
    indeksacji. Każda informacja, która wpływa do aplikacji, nie jest po prostu tam zapisywana.
    Specjalnie skonfigurowane indeksy pozwalają na określenie tego, co jest istotne. Ale nie jest
    to konieczne. Jedną z wartych wspomnienia funkcji, jest automatyczna detekcja typów danych, na ich
    podstawie. Wspomniane indeksy są tworzone automatycznie, aby jak najszybciej można było
    wykorzystać możliwości oferowane przez \textbf{ElasticSearch}.
    
    \textbf{Multitenancy}, cecha, wyróżnik chmur obliczeniowych, jest bardzo łatwo do osiągnięcia
    dzięki grupowania i indeksacji w \textbf{ElasticSearch}. Grupa idealnie oddaje koncepcję tenanta.
    Wszystkie dane dla niego zgromadzone mogą zostać odszukane, poprzez wskazania indeksu. Jest on
    tożsamy z grupą, a w dalszej kolejności tenantem.
    
    Jest to także rozwiązania łatwo skalowalne. \textbf{ElasticSearch} posiada ten mechanizm wbudowany.
    Innymi słowy każda nowo uruchomiana instancja poszukuje w sieci, w której działa, innych instancji
    samoistnie budując klaster złożony z lidera oraz replik. 
    Lider odpowiedzialny jest za koordynacją pracy klastra i może również przechowywać dane.
    Repliki stanowią odzwierciedlenie stanu lidera. Jeśli którakolwiek z instancji zostanie wyłączona z klastra,
    potrafi się o samodzielnie przebudować. Dane przesyłane są między replikami. Ostatecznie uzyskuje się
    stan równowagi. Utracenie lidera nie stanowi przeszkody dla poprawnego działania. Pozostałe maszyny 
    negocjują wybór nowego i całość jest ponownie gotowa do świadczenia pełnej funkcjonalności w relatywnie
    krótkim czasie.

\subsection{Kibana}
    Ostatni z elementów \textbf{ELKStack}. Zadaniem \textbf{Kibana} jest dostarczyć graficznego interfejsu
    użytkownika dla \textbf{ElasticSearch}. Z poziomu aplikacji dostępnej przez przeglądarkę możliwe
    jest przeglądania wszystkich danych zgromadzonych przez \textbf{Logstash} z użyciem tradycyjnych
    metod przeszukiwania danych:
    \begin{itemize}
        \item filtrowanie,
        \item kwerendy, w tym wypadku pełnotekstowe,
        \item generowanie wykresów,
        \item kolaboracja poprzez specjalne obiekty zwane \textit{dashboards}
    \end{itemize}
    
    \textbf{Kibana} jest szczególna, ponieważ z jednej strony wykorzystuje \textbf{ElasticSearch}
    jako źródło danych, a z drugiej strony jest on dla niej miejscem do zapisu własnej konfiguracji.
    
    \todo[inline] Omówić bardziej znane funkcje.