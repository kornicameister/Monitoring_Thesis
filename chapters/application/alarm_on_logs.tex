\subsection{Alarmy a logi}
\label{chapter:application:plans:alarm_on_logs}

Jednym z głównych założeń projektu \textbf{monasca} jest powiadamiania użytkownika o 
sytuacjach krytycznych zaistniałych w systemie. Funkcjonalność realizowana jest na podstawie
metryk, zbieranych z całego systemu. Przekroczenie, założonych przez administratora, wartości
krytycznych, skutkuje wygenerowaniem alarmu. Jednak idea ta, nie może zostać
tak łatwo przełożona na logi. 

    \subsubsection{Idea implementacji}
    \textbf{monasca-agent} jest programem, którego głównym zadaniem jest zbierania metryk z systemu, na którym
    został on zainstalowany. Podejście takie sprawdza się w pełni i właściwie jest jedynym możliwym rozwiązaniem,
    jeśli chodzi o monitoring stanu systemu operacyjnego, maszyny lub aplikacji na nim uruchomionych. Niekoniecznie,
    jest to jednak podejście właściwie dla podobnej analizy w przypadku logów. W rozwiązaniu typu \textbf{LaaS}, 
    jednym z celów jest przesłanie danych do centralnej lokalizacji - archiwizacja. W tym samym punkcie, lub
    przesłane dalej, są one podmiotem analizy. Implementacja będzie, przynajmniej koncepcyjne, zbliżona do \textbf{monasca-agent}.
    
    Możliwe są dwie implementacja:
    \begin{itemize}
        \item[dynamiczna] - potencjalny program miałby być zlokalizowany w za \textbf{monasca-log-transformer} (\ref{chapter:monasca:monasca_log_transformer}).
        Odbierałby on logi ze wspomniane programu, które wysyłane byłyby na ten sam lub inny \textbf{topic} i poddawał je prostej analizie. 
        \item[statyczna] - oparta o okresowe pobierania relatywnie małych ilości danych z bazy ElasticSearch (\ref{chapter:application:elkstack:elasticsearch}). 
        Zapytanie oczywiście byłoby zoptymalizowanie na pobranie jedynie tych rekordów, gdzie poziom logu osiągnął zadaną przez operatora wartość.
    \end{itemize}
    
    \subsubsection{Analiza dynamiczna w strumieniu}
    \label{chapter:application:plans:alarm_on_logs:streaming}
    Jest to alternatywa, która idealnie wpisuje się w rzeczywisty charakter danych, jakimi są logami. Ponadto, jest to rozwiązanie, w którym
    wyeliminowana zostaje konieczność iterowanie po zbiorach danych, których wielkość może znacząco spowalniać proces analizy. W programowaniu,
    pętle są najczęściej wskazywane jako najwolniejsze elementy aplikacji, chociaż oczywiście momentami nie da się ich uniknąć. Ponadto, nie ma 
    konieczności także okresowego pobierania danych, co znacząco wydłuża czas odpowiedzi program. Pod uwagę należy jednak wziąć pewne punkty wymagania,
    które musiałyby być spełnione, a które są wyznacznikami przetwarzania w czasie rzeczywistym.
    \begin{itemize}
        \item[ciągły przepływ] - dane, zgodnie z filozofią przetwarzania strumieniowego, muszą przepływać przez aplikację. Wynika z tego, że szybkość
        przetwarzania jest tutaj nadrzędnym celem do osiągnięcia,
        \item[przeciwdziałania strumieniowi] - mimo, że brzmi to jak jeden z antywzorców programowania,
        \footnote{\textbf{Fighting against framework} - próba celowego wykorzystania biblioteki lub wzorca do zadań, do których nie został on zaprojektowany},
        jest to ważna cecha przetwarzania danych w czasie rzeczywistym. Podobnie, jak w rzeczywistym potoku, dane, jak ryby, mogą pływać ławicami, ale mogą
        również napływać pojedynczo, w określonych, i co ważne - nieregularnych, odstępach czasu. Ważne jest więc, aby nie blokować programu, w oczekiwaniu
        na dane. Rozwiązaniem jest zastosowanie mechanizmów takich jak okna przesuwne, których działania powinno się zakończyć po określonym upływie czasu lub
        aktywne nasłuchiwania na pojawiające się dane na wejściu. Szczęśliwie, istnieje odpowiednie biblioteki, które realizuję funkcję aktywnego oczekiwania,
        i wznawiają właściwy wątek programu, jeśli takowa wiadomość nadejdzie. Z drugiej strony, dane mogą napływać dużo szybciej niż aplikacja jest w stanie
        je przetworzyć. W tym wypadku, konieczne może okazać się minimalne buforowanie danych lub przetwarzanie napływających informacji w sposób równoległy.
        \item[powtarzalne rezultaty] - aby silnik alarmujący na podstawie logów generował powtarzalne i, przede wszystkim, 
        spójne rezultaty konieczne jest, aby dane analizować, nie tyle pod kątem poziomu logu, ale także pod względem czasu w jakim został on wygenerowany. 
        Jest to szczególnie istotne, ponieważ system generujący alarm na podstawie tylko jednego punktu wykraczającego poza skalę byłby 
        narażony na zgłaszanie zbyt dużej liczby fałszywych alarmów. Innymi słowy, jeśli system wykryje n-krotne powtórzenie 
        niepoprawnej wartości, powinien wtedy, i tylko wtedy wygenerować alarm. 
        Co prawda, w przypadku logów, omówiona sytuacja, nie zawsze musi być prawdą. Jednostkowe przekroczenie zużycia procesora, nie jest sytuacją anormalną.
        Niemniej, tylko jeden błąd zalogowany przez aplikację o krytycznym znaczeniu może oznaczać błąd. Ale, jeśli program zapisuje w logach jedynie ostrzeżenia,
        dopiero pewna ich liczba, zaobserwowana w, przykładowo, ostatnich 10 minutach działania aplikacji, może zostać uznana, ze problem. Jeśli dane, nie byłyby
        przeglądane pod kątem czasu, w którym powstał wpis, rezultaty analizy byłyby nierzetelne. 10 krotne zaobserwowania wartości \textbf{WARNING} mogłoby tak naprawdę
        pochodzić z próbek wygenerowanych w ciągu ostatnich 30 minutach, a sam fakt, zgłoszenia błędu spowodowany tym, że dotarły one do silnika analizującego
        w oknie czasowym o długości co najwyżej 10 minut \cite{8_requirements_of_real_time_processing}.
        \item[grupowanie] - nie jest to problem, jeśli analizuje się zbiór danych. W danej próbce kontrolnej, dane można łatwo posortować i na jej podstawie 
        wygenerować pewien wynik. Sytuacja komplikuje się dla danych przetwarzanych w potoku. Potencjalny silnik alarmujący otrzymywałby dane pochodzącego z,
        potencjalnie, nieskończonej, liczby różnych programów. Operator, mógłby tak skonfigurować alarmy, aby wykrywać 5 krotne pojawienie się wartości \textbf{ERROR}
        dla aplikacji A, w ciągu 5 minut. Innymi słowy, aplikacja analityczny, musiałaby zapamiętać pojawienia się pierwszego wystąpienia problematycznej wartości.
        Kolejny wymogiem byłoby monitorowanie czasu, jaki upłynął od momentu zauważania wartości po raz pierwszy. Ostatecznie, wygenerowania odpowiedniego,
        alarmu powinna ograniczać się do tych wartości, które wykryto jedynie dla konkretnej aplikacji.
    \end{itemize}
    
    \subsubsection{Analiza tradycyjna na zbiorze}
    \label{chapter:application:plans:alarm_on_logs:bulk}
    Rozwiązanie alternatywne do omówionego w \ref{chapter:application:plans:alarm_on_logs:streaming}. Rozwiązuje ono wszystkie, przedstawione
    wyżej wymaganie, jakie musiałby spełnić silnik alarmujący, łącznie z grupowaniem. Aplikacja mogłaby pobierać dane, w określonym oknie czasowym,
    ze wskazaniem na poziomu logów oraz aplikacje. Jeśli w odebranych wynikach znalazłyby się odpowiednia
    ilość rekordów, gdzie wartości byłyby przekroczone, byłby to sygnał do wygenerowania alarmu. Niemniej, rozwiązanie tego typu, cechuje się
    narzutem wynikającym z:
    \begin{itemize}
        \item konieczność wygenerowania żądania, otrzymania odpowiedzi i ostatecznie rozpakowania danych do postaci, w której mogłyby zostać
        przetworzone,
        \item program nie może pobierać danych zbyt szybko. Pomijając fakt, że mogłoby to negatywnie wpłynąć na jej wydajność bazy danych, która
        musiałaby przeznaczyć więcej zasobów, aby zrealizować żądanie, wynikałoby to również z konieczności zakończenie przetwarzania już
        otrzymanych danych,
        \item program operowałby na zbiorze, o zmiennej wielkości. Nie byłoby, oczywiście, problematyczne zweryfikowanie małego lub pustego zbioru,
        niemniej otrzymanie dużej ilości rekordów, przełożyłoby się na łączne (czas przesłania, przetworzenia do zrozumiałego formatu, iteracji)
        wydłużenia czasu operacyjnego zaszytego algorytmu. 
    \end{itemize}
    W środowisku, gdzie konieczne jest, szybkie reagowania na ciągle zmieniające się warunki, jakiekolwiek opóźnienia są same w sobie, źródłem
    problemu. Niemniej rozwiązanie to, ma tę szczególną zaletę, że opiera się na istniejącej bazie danych o rozbudowanym API, zdolnym
    zwrócić dane w kompaktowej formie.
    
    Poboczną zaletą obu rozwiązań, jeśli wybranego by oddzielny topic przeznaczony jedynie dla programów analizujących dane, byłoby wykreowania
    swoistej tablicy trasowania (działającej na zasadzie zbliżonej do tej znanej z urządzeń sieciowych), co w dalszej kolejności mogłoby
    służyć do rozdysponowania przekształconych logów do różnych punktów docelowych, gdzie stałyby się one podmiotem całkowicie odrębnych operacji.
    Program miałby za zadania analizować zawartość przesłanych logów. Jeśli poziom, którekolwiek z nich przekroczył by zadaną (skonfigurowaną przez
    administratora lub operatora) wartość, aplikacja tworzyłaby odpowiednią metrykę.
    Obie z implementacji zostały zaproponowane przez autora pracy dyplomowej jako przeciwne do siebie, łącznie z ich zaletami oraz wadami.

    \subsubsection{Metryki nieciągłe}
    W obu przypadkach, niezależnie od przyjętej implementacji, problem jest przedstawienie zebranych danych w postaci metryk - czyli wartości
    numerycznych. W przeciwieństwie do danych dotyczących, chociażby, zużycia przestrzeni dyskowej lub, podlegającemu większej
    fluktuacji, obciążeniu procesora, logi stanowią element zbioru metryk nieciągłych \footnote{Sparse metric - metryki okresowe,
    zwracające wartości w nieregularnych odstępach czasu}. Częścią rozwiązania problemu, byłaby modyfikacja aplikacji \textbf{monasca-thresh},
    odpowiedzialnej za faktyczne generowania alarmów. Na obecną chwilę, jeśli dla danej metryki, brak jest danych, monitor zdefiniowany
    dla danego alarmu powoduje jego wejście w stan \textbf{UNDETERMINED}. W przypadku logów, nie jest to jednak sytuacja oczekiwana. 
    Przykładowo, brak wystąpień błędów dla danej aplikacji, powinien faktycznie oznaczać, że aplikacja lub system pracuje poprawnie i
    potencjalny operator nie ma powodów do niepokoju. Innymi słowy, odpowiadałoby to stanowi \textbf{OK}. Dopiero, co najmniej jedna wartość
    wykraczająca poza ramy danego alarmu, powinna spowodować przejście alarmu w stan wysoki oraz wygenerowanie powiadomienia lub powiadomień.