\section{monasca-log-transformer - proste transformacje}
\label{chapter:monasca:monasca_log_transformer}

\textbf{monasca-log-transformer} jest komponentem znajdującym się bezpośrednio \linebreak za \textbf{monasca-log-api}. Jest to ponownie specjalnie skonfigurowana instancja
\textbf{Logstash} (\ref{chapter:application:elkstack}). Nasłuchuje ona na \textbf{topic}'ach w oczekiwaniu
na wiadomości, które \textbf{monasca-log-api} na nie wysyła. 
\newglossaryentry{kafka_topic}{
    name={Topic},
    description={Topic jest w Kafce tożsamy z grupą. Podczas gdy jedna aplikacja wpisuje dane do wskazanej grupy, inne mogą czytać wiadomości zaadresowane
        do tej grupy. Możliwe jest również czytanie z większej ilości grup}
    }

Po odebraniu wiadomości transformer ma zadanie dokonać, zgodnie z nazwą transformacji, otrzymanego 
logu. Na obecną chwilę jedyną operacją, którą ma za zadanie wykonać jest analiza rekordu pobranego z logu,
aby wykryć jego poziom. Implementacja opiera się na wyrażeniu regularnym, którym objęty jest wycinek
całego logu. W wyniku działania wyrażenia regularnego do struktury całej wiadomości dodawana jest informacja
o poziomie logu. 

\begin{listing}
    \inputminted[
    fontfamily=monospace,
    fontsize=\scriptsize,
    obeytabs=true,
    samepage=true
    ]{rb}{listingings/transformer.conf}
    \caption[Przykładowa konfiguracja monasca-log-transformer]{
        Przykładowa konfiguracja monasca-log-transformer, źródło: \url{https://raw.githubusercontent.com/FujitsuEnablingSoftwareTechnologyGmbH/ansible-monasca-elkstack/master/templates/transformer.conf.j2}}
    \label{chapter:monasca:monasca_log_transformer:configuration_code}
\end{listing}

Listing \ref{chapter:monasca:monasca_log_transformer:configuration_code} wycinek kodu przedstawia
sekcję filtracji programu Logstash. Zawarty w niej kod Ruby oraz dopasowanie Grok pozwala na 
pobranie poziomu logu oraz znormalizowanie jej do spójnej wartości. 