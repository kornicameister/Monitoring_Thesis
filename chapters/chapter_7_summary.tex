\chapter[Podsumowanie]{Podsumowanie}
\label{chapter:summary}

Celem pracy, jak zostało to wcześniej przedstawione, było nawiązanie do monitorowania aplikacji, 
rozwijanych w chmurze, a więc i tym samym weryfikowania stanu maszyn (wirtualnych lub fizycznych) poprzez
okresowe pobieranie metryk, określających ich stan. Dużą rolą w tym aspekcie odgrywają także logi, czyli
informacja zapisywane przez program do, jak to się najczęściej dzieje, plików tekstowych zlokalizowanych
na tym samym hoście. Logi pozwalają zobrazować przebieg działania programu i tym różnią się od metryk, że
wgląd w ich zawartość daje nieprzerwany obraz, podobny do taśmy filmowej, w ciągle zmieniający się stan, w jakim
aplikacja w danym momencie się znajduje. 

Nawiązując do tematu pracy dyplomowej - \textbf{Zbierania, analiza oraz wnioskowanie - znaczenie logów w chmurach
    obliczeniowych}, nie udało się osiągnąć wszystkich zamierzonych celów. Autor, dzięki pracy całego zespołu oraz
społeczności projektu \textbf{monasca}, miał szansę i z sukcesem wprowadził oraz zaimplementował komponenty
odpowiedzialne przede wszystkim za zbieranie logów oraz wizualną ich analizę. Warto nadmienić, że udział autora
miał zarówno charakter pośredni (ocena kodu, akceptacja zmian) oraz bezpośredni (implementacja), a same zmiany
miały wysoce rozproszony charakter. Innymi słowy, fakt że nowe funkcji stały się częścią końcowego produktu, 
odzwierciedlony był w więcej niż jednym projekcie, wchodzącym w jego skład. Właśnie z tego powodu, niemożliwe było
skupienie się, na płaszczyźnie praktycznej, na zagadnieniu jakim jest analiza logów, oprócz udziału w kreacji
koncepcji alarmów na logach (\ref{chapter:application:plans:alarm_on_logs}). Dodatkowo, brak czasu, podyktowany był ciągle
napływający zgłoszeniami od zespołów zlokalizowanych w innych placówkach Fujitsu, zajmujących się kwestiami testów
funkcjonalnych oraz bezpieczeństwa.

Pomimo to, autor pracy dyplomowej, odniósł sukces, będąc częścią lub autorem wielu zmian, z których wiele dotyczyło
właściwych funkcji, a część stanowiła zadania poboczne, wspierające te wcześniej wymienione. Niestety z uwagi
na charakter części praktycznej, nie wszystkie mogły zostać w pracy dyplomowej omówione i przedstawione (\ref{chapter:application:plans}). Ponadto udało się uzyskać teoretyczny wgląd w całość procesu jakim
jest monitorowania oraz praca z logami dla systemów informatycznych, głównie na poziomie anatomii metryk, alarmów.
W kontekście logi, teoretyczne informacje dotyczące procesu efektywnego zbierania, normalizacji oraz archiwizacji, 
również zostały uzyskane. Dzięki nim, autor pracy magisterskiej, ma szansę na lepsze rozumienie stawianych
wymagań, a implementowane przez niego zmiany, będą charakteryzować się wyższym poziomie jakości. Okazuje się bowiem, że
zadania, tak z pozoru trywialne, jak przesyłania kolejnych rekordów z informacji, pobranymi z logów, aby ostatecznie
zaprezentować je użytkownikowi, są nad wyraz skomplikowane, jeśli we wszystkim obecny jest wysoce rozbudowany
i rozproszony system informatyczny.